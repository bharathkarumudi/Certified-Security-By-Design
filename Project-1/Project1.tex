\documentclass{report}
\title{Project 1}
\author{Bharath Karumudi}
\date{\today}

%%........................................%%
%% Loading the Packages 
%%........................................%%

\usepackage{634format}
\usepackage{enumerate}
\usepackage{listings}
\usepackage{textcomp}
\usepackage{amsmath}
\usepackage{hyperref}
\usepackage{holtex}
\usepackage{holtexbasic}
\input{commands}
%.........................................%%
%.........................................%%

\begin{document}

\lstset{language=ML}

\maketitle{}

\begin{abstract}

  This project is to demostrate the basic functional programming skills
  using the tools and techniques - \LaTeX{}, AcuTeX, emacs and ML. Each
  chapter documents the given problems with a structure of:
 
  \begin{itemize}
  \item Problem Statement
  \item Relevant Code
  \item Test Cases
  \end{itemize}

\end{abstract}

\begin{acknowledgments}
  Professor Marvine Hamner and Professor Shiu-Kai Chin who taught the
  Certified Security By Design.
\end{acknowledgments}

\tableofcontents{}

\chapter{Executive Summary}
\label{cha:executive-summary}
\textbf{All the requirements for this project are statisfied specifically,}

\begin{description}
\item[Contents] \ \\
Our report has the following content:
\begin{enumerate}
 \item Chapter~\ref{cha:executive-summary}: Executive Summary
 \item Chapter~\ref{cha:exercise-2.5.1}: Exercise 2.5.1
   \begin{enumerate}
    \item Section~\ref{sec:problem-statement} Problem Statement
    \item Section~\ref{sec:relevant-code} Relevant Code
    \item Section~\ref{sec:test-cases} Test Cases
    \item Section~\ref{sec:test-results} Test Results
   \end{enumerate}
 \item Chapter~\ref{cha:exercise-3.4.1} Exercise 3.4.1
   \begin{enumerate}
    \item Section~\ref{sec:problem-statement-1} Problem Statement
    \item Section~\ref{sec:relevant-code-1} Relevant Code
    \item Section~\ref{sec:test-results-1} Test Results
   \end{enumerate}
 \item Chapter~\ref{cha:exercise-3.4.2} Exercise 3.4.2
   \begin{enumerate}
    \item Section~\ref{sec:problem-statement-2} Problem Statement
    \item Section~\ref{sec:relevant-code-2} Relevant Code
    \item Section~\ref{sec:test-results-2} Test Results
   \end{enumerate}
\end{enumerate}
\item[Reproducibility in ML and \LaTeX{}] \ \\
 Our ML and \LaTeX{} source files compile with no errors.
\end{description}

\chapter{Exercise 2.5.1}
\label{cha:exercise-2.5.1}

\section{Problem Statement}
\label{sec:problem-statement}

In this exercise we are to define the following functions in ML:
\begin{align*}
 timesPlus\; x\; y &= (x*y,\; x+y)\\
\end{align*}

\section{Relevant Code}
\label{sec:relevant-code}

\lstset{frameround=tttt}
\begin{lstlisting}[frame=tRBL]
 fun timesPlus x y = (x*y, x+y);
\end{lstlisting}

\section{Test Cases}
\label{sec:test-cases}

The required test cases are:
\begin{lstlisting}[frame = tRBL ]
(****************************************************************************)
(* Test Cases*)
(****************************************************************************)
timesPlus 100 27;
timesPlus 10 26;
timesPlus 1 25;
timesPlus 2 24;
timesPlus 30 23;
timesPlus 50 200;
\end{lstlisting}


\section{Test Results}
\label{sec:test-results}

\setcounter{sessioncount}{0}
\begin{session}
  \begin{scriptsize}
\begin{verbatim}

> > > > val timesPlus = fn: int -> int -> int * int
> 
> timesPlus 100 27;
val it = (2700, 127): int * int
> timesPlus 10 26;
val it = (260, 36): int * int
> timesPlus 1 25;
val it = (25, 26): int * int
> timesPlus 2 24;
val it = (48, 26): int * int
> timesPlus 30 23;
val it = (690, 53): int * int
> timesPlus 50 200;
val it = (10000, 250): int * int
> 
>
\end{verbatim}    
  \end{scriptsize}
\end{session}


\chapter{Exercise 3.4.1}
\label{cha:exercise-3.4.1}

\section{Problem Statement}
\label{sec:problem-statement-1}

In this exercise, we will be solving the pattern matching:

\begin{align*}
&val\; listA\; =\; [(0,"Alice"),\; (1,"Bob"),\; (3,"Carol"),\; (4,"Dan")];\\
&val\; elB::listB\; =\; listA;\\
&val\; (e1C1,e1C2)\; =\; elB;\\
&val\; [e1C3,e1C4,e1C5]\; =\; listB;\\
\end{align*}

\section{Relevant Code}
\label{sec:relevant-code-1}

\lstset{frameround=tttt}
\begin{lstlisting}[frame=tRBL]

val listA = [(0,"Alice"), (1,"Bob"), (3,"Carol"),(4,"Dan")];

val elB::listB = listA;
val (e1C1,e1C2) = elB;
val [e1C3, e1C4, e1C5] = listB;
\end{lstlisting}

\section{Test Results}
\label{sec:test-results-1}

\setcounter{sessioncount}{0}
\begin{session}
  \begin{scriptsize}
\begin{verbatim}

> > > > val listA = [(0, "Alice"), (1, "Bob"), (3, "Carol"), (4, "Dan")]:
   (int * string) list
> val elB = (0, "Alice"): int * string
val listB = [(1, "Bob"), (3, "Carol"), (4, "Dan")]: (int * string) list
> val e1C1 = 0: int
val e1C2 = "Alice": string
> val e1C3 = (1, "Bob"): int * string
val e1C4 = (3, "Carol"): int * string
val e1C5 = (4, "Dan"): int * string
> 
 
\end{verbatim}
  \end{scriptsize}
\end{session}

\chapter{Exercise 3.4.2}
\label{cha:exercise-3.4.2}

\section{Problem Statement}
\label{sec:problem-statement-2}
In this exercise we will evaluate the following assignment statements and provide the reason in case if there are any errors.


\begin{align*} 
&val\; (x1,x2,x3)\; =\; (1,true,"Alice");\\
&val\; pair1\; =\; (x1,x3);\\
&val\; list1\; =\; [0,x1,2];\\
&val\; list2\; =\; [x2,x1];\\
&val\; list3\; =\; (1\; ::\; [x3]);\\ 
\end{align*}

\section{Relevant Code}
\label{sec:relevant-code-2}

\lstset{frameround=tttt}
\begin{lstlisting}[frame=tRBL]

val (x1,x2,x3) = (1,true,"Alice");
val pair1 = (x1,x3);
val list1 = [0,x1,2];
val list2 = [x2,x1];
val list3 = (1 :: [x3]);

\end{lstlisting}

\section{Test Results}
\label{sec:test-results-2}

\setcounter{sessioncount}{0}
\begin{session}
  \begin{scriptsize}
\begin{verbatim}

> > > > 
> val x1 = 1: int
val x2 = true: bool
val x3 = "Alice": string
> val pair1 = (1, "Alice"): int * string
> val list1 = [0, 1, 2]: int list
> poly: : error: Elements in a list have different types.
   Item 1: x2 : bool
   Item 2: x1 : int
   Reason:
      Can't unify bool (*In Basis*) with int (*In Basis*)
         (Different type constructors)
Found near [x2, x1]
Static Errors
> poly: : error: Type error in function application.
   Function: :: : int * int list -> int list
   Argument: (1, [x3]) : int * string list
   Reason:
      Can't unify int (*In Basis*) with string (*In Basis*)
         (Different type constructors)
Found near (1 :: [x3])
Static Errors
> 
 
\end{verbatim}
  \end{scriptsize}
\end{session}

\subsection{Explanation for Errors}
The errors occured in the statements are due to:
\begin{itemize}
\item val list2 = [x2, x1]; is due to creating a list with different types,
where x2 is a boolean and x1 is a integer. A list will take similar
data types.
\item val list3 = (1 :: [x3]); is due to creating a list with different types. 1 is a integer and x3 is a string type. HOL cannot create a list of two different data types.
\end{itemize}

\chapter{Appendix A: Exercise 2.5.1}
\label{cha:appendix-a:-exercise}

The following code is from the file ex-2-5-1.sml
\lstinputlisting{ML/ex-2-5-1.sml}


\chapter{Appendix B: Exercise 3.4.1}
\label{cha:appendix-b:-exercise}

The following code is from the file ex-3-4-1.sml
\lstinputlisting{ML/ex-3-4-1.sml}

\chapter{Appendix C: Exercise 3.4.2}
\label{cha:appendix-c:-exercise}

The following code is from the file ex-3-4-2.sml
\lstinputlisting{ML/ex-3-4-2.sml}

\end{document}

