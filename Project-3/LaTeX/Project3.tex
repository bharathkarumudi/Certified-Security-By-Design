\documentclass{report}
\title{Project 3}
\author{Bharath Karumudi}
\date{\today}

%%........................................%%
%% Loading the Packages 
%%........................................%%
\usepackage{634format}
\usepackage{enumerate}
\usepackage{listings}
\usepackage{amsmath}
\usepackage{hyperref}
\usepackage{holtex}
\usepackage{holtexbasic}
\usepackage{amssymb}
\input{commands}
\newcommand{\HOLchapterEightDate}{19 August 2019}
\newcommand{\HOLchapterEightTime}{21:11}
\begin{SaveVerbatim}{HOLchapterEightTheoremsconjSymThm}
\HOLTokenTurnstile{} \HOLFreeVar{p} \HOLSymConst{\HOLTokenConj{}} \HOLFreeVar{q} \HOLSymConst{\HOLTokenEquiv{}} \HOLFreeVar{q} \HOLSymConst{\HOLTokenConj{}} \HOLFreeVar{p}
\end{SaveVerbatim}
\newcommand{\HOLchapterEightTheoremsconjSymThm}{\UseVerbatim{HOLchapterEightTheoremsconjSymThm}}
\begin{SaveVerbatim}{HOLchapterEightTheoremsconjSymThmAll}
\HOLTokenTurnstile{} \HOLSymConst{\HOLTokenForall{}}\HOLBoundVar{p} \HOLBoundVar{q}. \HOLBoundVar{p} \HOLSymConst{\HOLTokenConj{}} \HOLBoundVar{q} \HOLSymConst{\HOLTokenEquiv{}} \HOLBoundVar{q} \HOLSymConst{\HOLTokenConj{}} \HOLBoundVar{p}
\end{SaveVerbatim}
\newcommand{\HOLchapterEightTheoremsconjSymThmAll}{\UseVerbatim{HOLchapterEightTheoremsconjSymThmAll}}
\begin{SaveVerbatim}{HOLchapterEightTheoremsproblemOneThm}
\HOLTokenTurnstile{} \HOLFreeVar{p} \HOLSymConst{\HOLTokenImp{}} (\HOLFreeVar{p} \HOLSymConst{\HOLTokenImp{}} \HOLFreeVar{q}) \HOLSymConst{\HOLTokenImp{}} (\HOLFreeVar{q} \HOLSymConst{\HOLTokenImp{}} \HOLFreeVar{r}) \HOLSymConst{\HOLTokenImp{}} \HOLFreeVar{r}
\end{SaveVerbatim}
\newcommand{\HOLchapterEightTheoremsproblemOneThm}{\UseVerbatim{HOLchapterEightTheoremsproblemOneThm}}
\newcommand{\HOLchapterEightTheorems}{
\HOLThmTag{chapter8}{conjSymThm}\HOLchapterEightTheoremsconjSymThm
\HOLThmTag{chapter8}{conjSymThmAll}\HOLchapterEightTheoremsconjSymThmAll
\HOLThmTag{chapter8}{problem1Thm}\HOLchapterEightTheoremsproblemOneThm
}

%.........................................%%
%.........................................%%

\begin{document}
 \lstset{language=ML}
 \maketitle{}

 \begin{abstract}
   This project is to demonstrate the capabilities of implementiong
   constructing and deconstructing HOL Terms using the tools and
   techniques - \LaTeX{}, AcuTeX, emacs and ML. 

   Each chapter documents the given problems with a structure of:
   \begin{enumerate}
   \item Problem Statement
   \item Relevant Code
   \item Execution Transcripts
   \item Explanation of results
   \end{enumerate}

 \end{abstract}


 \begin{acknowledgments}
  Professor Marvine Hamner and Professor Shiu-Kai Chin who taught the
  Certified Security By Design.
 \end{acknowledgments}

 \tableofcontents{}

 \chapter{Executive Summary}
 \label{cha:executive-summary}

\textbf{All the requirements for this project are statisfied specifically,}
 and by using HOL proved the below theorems:

\begin{description}
\item[Contents] \ \\
Our report has the following content:
\begin{enumerate}
 \item Chapter~\ref{cha:executive-summary}: Executive Summary
 \item Chapter~\ref{cha:exercise-7.3.1} Exercise 7.3.1
   \begin{enumerate}
    \item Section~\ref{sec:problem-statement-71} Problem Statement
    \item Section~\ref{sec:relevant-code-71} Relevant Code
     \item Section~\ref{sec:test-cases-71} Test Cases
    \item Section~\ref{sec:exec-transcr-71} Execution Transcripts
    \item Section~\ref{sec:explanation-results-71} Explanation of Results
   \end{enumerate}
 \item Chapter~\ref{cha:exercise-7.3.2} Exercise 7.3.2
   \begin{enumerate}
    \item Section~\ref{sec:problem-statement-72} Problem Statement
    \item Section~\ref{sec:relevant-code-72} Relevant Code
     \item Section~\ref{sec:test-cases-72} Test Cases
    \item Section~\ref{sec:exec-transcr-72} Execution Transcripts
    \item Section~\ref{sec:explanation-results-72} Explanation of Results
   \end{enumerate}
 \item Chapter~\ref{cha:exercise-7.3.3} Exercise 7.3.3
   \begin{enumerate}
    \item Section~\ref{sec:problem-statement-73} Problem Statement
    \item Section~\ref{sec:relevant-code-73} Relevant Code
    \item Section~\ref{sec:test-cases-73} Test Cases
    \item Section~\ref{sec:exec-transcr-73} Execution Transcripts
    \item Section~\ref{sec:explanation-results-73} Explanation of Results
   \end{enumerate}
\end{enumerate}
\end{description}

\textbf{Chapter 8:}
\begin{quote}
\HOLchapterEightTheorems
\end{quote}

Reproducibility in ML and \LaTeX{}: Our ML and \LaTeX{} source files compile with no errors.


%%------ Exercise 7 -------%%

\chapter{Exercise 7.3.1}
 \label{cha:exercise-7.3.1}
  
 \section{Problem Statement}
 \label{sec:problem-statement-71}

In this exercise we need to create a function \emph{andImp2Imp term}, which will take:
\begin{align*}
p\wedge q \subset r
\end{align*}

and results to:
\begin{align*}
p \subset q \subset r;
\end{align*}

\section{Relevant Code}
\label{sec:relevant-code-71}
\lstinputlisting{HOLCode/ex-7-3-1.sml}

\section{Test Cases}
\label{sec:test-cases-71}

The required test cases are:
\begin{lstlisting}[frame = tRBL ]
andImp2Imp ``(p/\q) ==> r``
\end{lstlisting}

\section{Execution Transcripts}
\label{sec:exec-transcr-71}

\setcounter{sessioncount}{0}
\begin{session}
  \begin{scriptsize}
\begin{verbatim}

---------------------------------------------------------------------
       HOL-4 [Kananaskis 11 (stdknl, built Sat Aug 19 09:30:06 2017)]

       For introductory HOL help, type: help "hol";
       To exit type <Control>-D
---------------------------------------------------------------------
> > > > # # # # # # # # # ** types trace now on
> *** Globals.show_assums now true ***
> # # # # # # # # # ** Unicode trace now off
> 
> # # # # # # # val andImp2Imp = fn: term -> term
> > 
> 
> andImp2Imp ``(p/\q) ==> r``;
val it =
   ``(p :bool) ==> (q :bool) ==> (r :bool)``:
   term
> 
\end{verbatim}
  \end{scriptsize}
\end{session}

\subsection{Explanation of Results}
\label{sec:explanation-results-71}
The above test results shows the test case has been passed.


%%---- Exercise 7.3.2 -----%

\chapter{Exercise 7.3.2}
\label{cha:exercise-7.3.2}

\section{Problem Statement}
\label{sec:problem-statement-72}
In this exercise, we have to create \emph{andImp2Imp term}, which takes the term 
\begin{align*}
p \subset q \subset r;
\end{align*}

and results to:
\begin{align*}
p\wedge q \subset r
\end{align*} and also should act as a reverse function for 7.3.1

\section{Relevant Code}
\label{sec:relevant-code-72}
\lstinputlisting{HOLCode/ex-7-3-2.sml}

\section{Test Cases}
\label{sec:test-cases-72}

The required test cases are:
\begin{lstlisting}[frame = tRBL ]
andImp2Imp ``(p/\q) ==> r``
impImpAnd ``p ==> q ==> r``;
impImpAnd(andImp2Imp ``(p/\q) ==> r``);
andImp2Imp(impImpAnd ``p==>q==>r``);
\end{lstlisting}


\section{Execution Transcripts}
\label{sec:exec-transcr-72}

\setcounter{sessioncount}{0}
\begin{session}
  \begin{scriptsize}
\begin{verbatim}

---------------------------------------------------------------------
       HOL-4 [Kananaskis 11 (stdknl, built Sat Aug 19 09:30:06 2017)]

       For introductory HOL help, type: help "hol";
       To exit type <Control>-D
---------------------------------------------------------------------
> > > > 
> # # # # # # # # # ** types trace now on
> *** Globals.show_assums now true ***
> # # # # # # # # # ** Unicode trace now off
> 
> # # # # # # # val andImp2Imp = fn: term -> term
> # # # # # # # # val impImpAnd = fn: term -> term
> > 
> val it =
   ``(p :bool) ==> (q :bool) ==> (r :bool)``:
   term
> val it =
   ``(p :bool) /\ (q :bool) ==> (r :bool)``:
   term
> val it =
   ``(p :bool) /\ (q :bool) ==> (r :bool)``:
   term
> val it =
   ``(p :bool) ==> (q :bool) ==> (r :bool)``:
   term
> 

\end{verbatim}
  \end{scriptsize}
\end{session}

\subsection{Explanation of Results}
\label{sec:explanation-results-72}

The above transcript shows the given test cases has been passed.

%%---- Exercise 7.3.3 -----%

\chapter{Exercise 7.3.3}
\label{cha:exercise-7.3.3}

\section{Problem Statement}
\label{sec:problem-statement-73}
In this exercise we have to create a function \emph{notExists term} which takes the term  $\neg \exists x.P(x)$ and returns $\forall x.\neg P(x)$.

\section{Relevant Code}
\label{sec:relevant-code-73}
\lstinputlisting{HOLCode/ex-7-3-3.sml}

\section{Test Cases}
\label{sec:test-cases-73}

The required test cases are:
\begin{lstlisting}[frame = tRBL ]
notExists ``~?z.Q z``;
\end{lstlisting}


\section{Execution Transcripts}
\label{sec:exec-transcr-73}

\setcounter{sessioncount}{0}
\begin{session}
  \begin{scriptsize}
\begin{verbatim}

---------------------------------------------------------------------
       HOL-4 [Kananaskis 11 (stdknl, built Sat Aug 19 09:30:06 2017)]

       For introductory HOL help, type: help "hol";
       To exit type <Control>-D
---------------------------------------------------------------------
> > > > 
> 
> # # # # # # # # # ** types trace now on
> *** Globals.show_assums now true ***
> # # # # # # # # # ** Unicode trace now off
> 
> # # # # # val notExists = fn: term -> term
> > 
> <<HOL message: inventing new type variable names: 'a>>
val it =
   ``!(z :'a). (Q :'a -> bool) z``:
   term
> 

\end{verbatim}
  \end{scriptsize}
\end{session}

\subsection{Explanation of Results}
\label{sec:explanation-results-73}

The above transcript shows the given tests has been passed.



%%------ Exercise 8.4.1 -------%%

 \chapter{Exercise 8.4.1}
 \label{cha:exercise-8.4.1}
  
 \section{Problem Statement}
 \label{sec:problem-statement-1}

In this exercise we need to prove the theorem:
\HOLTokenTurnstile{} \HOLFreeVar{p} \HOLSymConst{\HOLTokenImp{}} (\HOLFreeVar{p} \HOLSymConst{\HOLTokenImp{}} \HOLFreeVar{q}) \HOLSymConst{\HOLTokenImp{}} (\HOLFreeVar{q} \HOLSymConst{\HOLTokenImp{}} \HOLFreeVar{r}) \HOLSymConst{\HOLTokenImp{}} \HOLFreeVar{r}

\section{Relevant Code}
\label{sec:relevant-code-1}
\lstinputlisting{HOLCode/ex-8-4-1.sml}

\section{Execution Transcripts}
\label{sec:exec-transcr-1}

\setcounter{sessioncount}{0}
\begin{session}
  \begin{scriptsize}
\begin{verbatim}

---------------------------------------------------------------------
       HOL-4 [Kananaskis 11 (stdknl, built Sat Aug 19 09:30:06 2017)]

       For introductory HOL help, type: help "hol";
       To exit type <Control>-D
---------------------------------------------------------------------
> > > > # # # # # # # # # ** types trace now on
> *** Globals.show_assums now true ***
> # # # # # # # # # ** Unicode trace now off
> 
> 
> # # # # # # # # # # # val problem1Thm =
    [] |- (p :bool) ==> (p ==> (q :bool)) ==> (q ==> (r :bool)) ==> r:
   thm
> 
> > 
\end{verbatim}
  \end{scriptsize}
\end{session}

\subsection{Explanation of Results}
\label{sec:explanation-results-1}
The above results shows that theorem is proved.

%%------ Exercise 8.4.2 -------%%

 \chapter{Exercise 8.4.2}
 \label{cha:exercise-8.4.2}
  
 \section{Problem Statement}
 \label{sec:problem-statement-2}

In this exercise we need to prove the theorem:
\HOLTokenTurnstile{} \HOLSymConst{\HOLTokenForall{}}\HOLBoundVar{p} \HOLBoundVar{q}. \HOLBoundVar{p} \HOLSymConst{\HOLTokenConj{}} \HOLBoundVar{q} \HOLSymConst{\HOLTokenEquiv{}} \HOLBoundVar{q} \HOLSymConst{\HOLTokenConj{}} \HOLBoundVar{p}

\section{Relevant Code}
\label{sec:relevant-code-2}
\lstinputlisting{HOLCode/ex-8-4-2.sml}

\section{Execution Transcripts}
\label{sec:exec-transcr-2}

\setcounter{sessioncount}{0}
\begin{session}
  \begin{scriptsize}
\begin{verbatim}

---------------------------------------------------------------------
       HOL-4 [Kananaskis 11 (stdknl, built Sat Aug 19 09:30:06 2017)]

       For introductory HOL help, type: help "hol";
       To exit type <Control>-D
---------------------------------------------------------------------
> > > > # # # # # # # # # ** types trace now on
> *** Globals.show_assums now true ***
> # # # # # # # # # ** Unicode trace now off
> 
> # # # # # # # # val conj1Thm =
    [] |- (p :bool) /\ (q :bool) ==> q /\ p:
   thm
> > # # # # # # # # val conj2Thm =
    [] |- (q :bool) /\ (p :bool) ==> p /\ q:
   thm
> > # val conjSymThm =
    [] |- (p :bool) /\ (q :bool) <=> q /\ p:
   thm
> > > > 
*** Emacs/HOL command completed ***

> 

\end{verbatim}
  \end{scriptsize}
\end{session}

\subsection{Explanation of Results}
\label{sec:explanation-results-2}
The above results shows that theorem is proved.


%%------ Exercise 8.4.3 -------%%

 \chapter{Exercise 8.4.3}
 \label{cha:exercise-8.4.3}
  
 \section{Problem Statement}
 \label{sec:problem-statement-3}

In this exercise we need to prove the theorem:
\HOLTokenTurnstile{} \HOLFreeVar{p} \HOLSymConst{\HOLTokenImp{}} (\HOLFreeVar{p} \HOLSymConst{\HOLTokenImp{}} \HOLFreeVar{q}) \HOLSymConst{\HOLTokenImp{}} (\HOLFreeVar{q} \HOLSymConst{\HOLTokenImp{}} \HOLFreeVar{r}) \HOLSymConst{\HOLTokenImp{}} \HOLFreeVar{r}

\section{Relevant Code}
\label{sec:relevant-code-3}
\lstinputlisting{HOLCode/ex-8-4-3.sml}


\section{Execution Transcripts}
\label{sec:exec-transcr-3}

\setcounter{sessioncount}{0}
\begin{session}
  \begin{scriptsize}
\begin{verbatim}

---------------------------------------------------------------------
       HOL-4 [Kananaskis 11 (stdknl, built Sat Aug 19 09:30:06 2017)]

       For introductory HOL help, type: help "hol";
       To exit type <Control>-D
---------------------------------------------------------------------
> > > > # # # # # # # # # ** types trace now on
> *** Globals.show_assums now true ***
> # # # # # # # # # ** Unicode trace now off
> 
> # # # # # # # # val conj1Thm =
    [] |- (p :bool) /\ (q :bool) ==> q /\ p:
   thm
> > # # # # # # # # val conj2Thm =
    [] |- (q :bool) /\ (p :bool) ==> p /\ q:
   thm
> > # val conjSymThm =
    [] |- (p :bool) /\ (q :bool) <=> q /\ p:
   thm
> > val conjSymThmAll =
    [] |- !(p :bool) (q :bool). p /\ q <=> q /\ p:
   thm
> > # > 
*** Emacs/HOL command completed ***

> 
\end{verbatim}
  \end{scriptsize}
\end{session}

\subsection{Explanation of Results}
\label{sec:explanation-results-3}
The above results shows that theorem is proved.

\chapter{Appendix A: Chapter 7}
\label{cha:appendix-a:chapter7}

The following code is from the file chapter7Answers.sml
\lstinputlisting{../ML/chapter7Answers.sml}

\chapter{Appendix B: Chapter 8}
\label{cha:appendix-a:chapter8}

The following code is from the file project3bScript.sml
\lstinputlisting{../HOL/chapter8Script.sml}

\end{document}