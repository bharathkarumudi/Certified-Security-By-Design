\documentclass{report}
\title{Project 2}
\author{Bharath Karumudi}
\date{\today}

%%........................................%%
%% Loading the Packages 
%%........................................%%
\usepackage{634format}
\usepackage{enumerate}
\usepackage{listings}
\usepackage{amsmath}
\usepackage{hyperref}
\usepackage{holtex}
\usepackage{holtexbasic}
\usepackage{amssymb}
% =====================================================================
%
% Macros for typesetting the HOL system manual
%
% =====================================================================

% ---------------------------------------------------------------------
% Abbreviations for words and phrases
% ---------------------------------------------------------------------

\newcommand\TUTORIAL{{\footnotesize\sl TUTORIAL}}
\newcommand\DESCRIPTION{{\footnotesize\sl DESCRIPTION}}
\newcommand\REFERENCE{{\footnotesize\sl REFERENCE}}
\newcommand\LOGIC{{\footnotesize\sl LOGIC}}
\newcommand\LIBRARIES{{\footnotesize\sl LIBRARIES}}

\newcommand{\bs}{\texttt{\char'134}} % backslash
\newcommand{\lb}{\texttt{\char'173}} % left brace
\newcommand{\rb}{\texttt{\char'175}} % right brace
\newcommand{\td}{\texttt{\char'176}} % tilde
\newcommand{\lt}{\texttt{\char'74}} % less than
\newcommand{\gt}{\texttt{\char'76}} % greater than
\newcommand{\dol}{\texttt{\char'44}} % dollar
% double back quotes ``
\newcommand{\dq}{\texttt{\char'140\char'140}}
%These macros were included by slind:

\newcommand{\holquote}[1]{\dq#1\dq}

\def\HOL{{\small HOL}}
\def\holn{\HOL}  % i.e. hol n(inety-eight), no digits in
                 % macro names is a bit of a pain; deciding to do away
                 % with hol98 nomenclature means that we just want to
                 % write HOL for hol98.
\def\holnversion{Kananaskis-7}
\def\holnsversion{Kananaskis~7} % version with space rather than hyphen
\def\LCF{{\small LCF}}
\def\LCFLSM{{\small LCF{\kern-.2em}{\normalsize\_}{\kern0.1em}LSM}}
\def\PPL{{\small PP}{\kern-.095em}$\lambda$}
\def\PPLAMBDA{{\small PPLAMBDA}}
\def\ML{{\small ML}}
\def\holmake{\texttt{Holmake}}

\newcommand\ie{\mbox{i{.}e{.}}}
\newcommand\eg{\mbox{e{.}g{.}}}
\newcommand\viz{\mbox{viz{.}}}
\newcommand\adhoc{\mbox{\it ad hoc}}
\newcommand\etal{{\it et al.\/}}
\newcommand\etc{\mbox{etc{.}}}

% ---------------------------------------------------------------------
% Simple abbreviations and macros for mathematical typesetting
% ---------------------------------------------------------------------

\newcommand\fun{{\to}}
\newcommand\prd{{\times}}

\newcommand\conj{\ \wedge\ }
\newcommand\disj{\ \vee\ }
\newcommand\imp{ \Rightarrow }
\newcommand\eqv{\ \equiv\ }
\newcommand\cond{\rightarrow}
\newcommand\vbar{\mid}
\newcommand\turn{\ \vdash\ }
\newcommand\hilbert{\varepsilon}
\newcommand\eqdef{\ \equiv\ }

\newcommand\natnums{\mbox{${\sf N}\!\!\!\!{\sf N}$}}
\newcommand\bools{\mbox{${\sf T}\!\!\!\!{\sf T}$}}

\newcommand\p{$\prime$}
\newcommand\f{$\forall$\ }
\newcommand\e{$\exists$\ }

\newcommand\orr{$\vee$\ }
\newcommand\negg{$\neg$\ }

\newcommand\arrr{$\rightarrow$}
\newcommand\hex{$\sharp $}

\newcommand{\uquant}[1]{\forall #1.\ }
\newcommand{\equant}[1]{\exists #1.\ }
\newcommand{\hquant}[1]{\hilbert #1.\ }
\newcommand{\iquant}[1]{\exists ! #1.\ }
\newcommand{\lquant}[1]{\lambda #1.\ }

\newcommand{\leave}[1]{\\[#1]\noindent}
\newcommand\entails{\mbox{\rule{.3mm}{4mm}\rule[2mm]{.2in}{.3mm}}}

% ---------------------------------------------------------------------
% Font-changing commands
% ---------------------------------------------------------------------

\newcommand{\theory}[1]{\hbox{{\small\tt #1}}}
\newcommand{\theoryimp}[1]{\texttt{#1}}

\newcommand{\con}[1]{{\sf #1}}
\newcommand{\rul}[1]{{\tt #1}}
\newcommand{\ty}[1]{\textsl{#1}}

\newcommand{\ml}[1]{\mbox{{\def\_{\char'137}\texttt{#1}}}}
\newcommand{\holtxt}[1]{\ml{#1}}
\newcommand\ms{\tt}
\newcommand{\s}[1]{{\small #1}}

\newcommand{\pin}[1]{{\bf #1}}
\def\m#1{\mbox{\normalsize$#1$}}

% ---------------------------------------------------------------------
% Abbreviations for particular mathematical constants etc.
% ---------------------------------------------------------------------

\newcommand\T{\con{T}}
\newcommand\F{\con{F}}
\newcommand\OneOne{\con{One\_One}}
\newcommand\OntoSubset{\con{Onto\_Subset}}
\newcommand\Onto{\con{Onto}}
\newcommand\TyDef{\con{Type\_Definition}}
\newcommand\Inv{\con{Inv}}
\newcommand\com{\con{o}}
\newcommand\Id{\con{I}}
\newcommand\MkPair{\con{Mk\_Pair}}
\newcommand\IsPair{\con{Is\_Pair}}
\newcommand\Fst{\con{Fst}}
\newcommand\Snd{\con{Snd}}
\newcommand\Suc{\con{Suc}}
\newcommand\Nil{\con{Nil}}
\newcommand\Cons{\con{Cons}}
\newcommand\Hd{\con{Hd}}
\newcommand\Tl{\con{Tl}}
\newcommand\Null{\con{Null}}
\newcommand\ListPrimRec{\con{List\_Prim\_Rec}}


\newcommand\SimpRec{\con{Simp\_Rec}}
\newcommand\SimpRecRel{\con{Simp\_Rec\_Rel}}
\newcommand\SimpRecFun{\con{Simp\_Rec\_Fun}}
\newcommand\PrimRec{\con{Prim\_Rec}}
\newcommand\PrimRecRel{\con{Prim\_Rec\_Rel}}
\newcommand\PrimRecFun{\con{Prim\_Rec\_Fun}}

\newcommand\bool{\ty{bool}}
\newcommand\num{\ty{num}}
\newcommand\ind{\ty{ind}}
\newcommand\lst{\ty{list}}

% ---------------------------------------------------------------------
% \minipagewidth = \textwidth minus 1.02 em
% ---------------------------------------------------------------------

\newlength{\minipagewidth}
\setlength{\minipagewidth}{\textwidth}
\addtolength{\minipagewidth}{-1.02em}

% ---------------------------------------------------------------------
% Environment for the items on the title page of a case study
% ---------------------------------------------------------------------

\newenvironment{inset}[1]{\noindent{\large\bf #1}\begin{list}%
{}{\setlength{\leftmargin}{\parindent}%
\setlength{\topsep}{-.1in}}\item }{\end{list}\vskip .4in}

% ---------------------------------------------------------------------
% Macros for little HOL sessions displayed in boxes.
%
% Usage: (1) \setcounter{sessioncount}{1} resets the session counter
%
%        (2) \begin{session}\begin{verbatim}
%             .
%              < lines from hol session >
%             .
%            \end{verbatim}\end{session}
%
%            typesets the session in a numbered box.
% ---------------------------------------------------------------------

\newlength{\hsbw}
\setlength{\hsbw}{\textwidth}
\addtolength{\hsbw}{-\arrayrulewidth}
\addtolength{\hsbw}{-\tabcolsep}
\newcommand\HOLSpacing{13pt}

\newcounter{sessioncount}
\setcounter{sessioncount}{0}

\newenvironment{session}{\begin{flushleft}
 \refstepcounter{sessioncount}
 \begin{tabular}{@{}|c@{}|@{}}\hline
 \begin{minipage}[b]{\hsbw}
 \vspace*{-.5pt}
 \begin{flushright}
 \rule{0.01in}{.15in}\rule{0.3in}{0.01in}\hspace{-0.35in}
 \raisebox{0.04in}{\makebox[0.3in][c]{\footnotesize\sl \thesessioncount}}
 \end{flushright}
 \vspace*{-.55in}
 \begingroup\small\baselineskip\HOLSpacing}{\endgroup\end{minipage}\\ \hline
 \end{tabular}
 \end{flushleft}}

% ---------------------------------------------------------------------
% Macro for boxed ML functions, etc.
%
% Usage: (1) \begin{holboxed}\begin{verbatim}
%               .
%               < lines giving names and types of mk functions >
%               .
%            \end{verbatim}\end{holboxed}
%
%            typesets the given lines in a box.
%
%            Conventions: lines are left-aligned under the "g" of begin,
%            and used to highlight primary reference for the ml function(s)
%            that appear in the box.
% ---------------------------------------------------------------------

\newenvironment{holboxed}{\begin{flushleft}
  \begin{tabular}{@{}|c@{}|@{}}\hline
  \begin{minipage}[b]{\hsbw}
% \vspace*{-.55in}
  \vspace*{.06in}
  \begingroup\small\baselineskip\HOLSpacing}{\endgroup\end{minipage}\\ \hline
  \end{tabular}
  \end{flushleft}}

% ---------------------------------------------------------------------
% Macro for unboxed ML functions, etc.
%
% Usage: (1) \begin{hol}\begin{verbatim}
%               .
%               < lines giving names and types of mk functions >
%               .
%            \end{verbatim}\end{hol}
%
%            typesets the given lines exactly like {boxed}, except there's
%            no box.
%
%            Conventions: lines are left-aligned under the "g" of begin,
%            and used to display ML code in verbatim, left aligned.
% ---------------------------------------------------------------------

\newenvironment{hol}{\begin{flushleft}
 \begin{tabular}{c@{}@{}}
 \begin{minipage}[b]{\hsbw}
% \vspace*{-.55in}
 \vspace*{.06in}
 \begingroup\small\baselineskip\HOLSpacing}{\endgroup\end{minipage}\\
 \end{tabular}
 \end{flushleft}}

% ---------------------------------------------------------------------
% Emphatic brackets
% ---------------------------------------------------------------------

\newcommand\leb{\lbrack\!\lbrack}
\newcommand\reb{\rbrack\!\rbrack}


% ---------------------------------------------------------------------
% Quotations
% ---------------------------------------------------------------------


%These macros were included by ap; they are used in Chapters 9 and 10
%of the HOL DESCRIPTION

\newcommand{\inds}%standard infinite set
 {\mbox{\rm I}}

\newcommand{\ch}%standard choice function
 {\mbox{\rm ch}}

\newcommand{\den}[1]%denotational brackets
 {[\![#1]\!]}

\newcommand{\two}%standard 2-element set
 {\mbox{\rm 2}}

%.........................................%%
%.........................................%%

\begin{document}
 \lstset{language=ML}
 \maketitle{}

 \begin{abstract}
   This project is to demonstrate the capabilities of functional
   programming using the tools and techniques - \LaTeX{}, AcuTeX,
   emacs and ML. Each chapter documents the given problems with a
   structure of:
   \begin{enumerate}
   \item Problem Statement
   \item Relevant Code
   \item Test Cases
   \item Execution Transcripts
   \item Explanation of results
   \end{enumerate}

 \end{abstract}


 \begin{acknowledgments}
  Professor Marvine Hamner and Professor Shiu-Kai Chin who taught the
  Certified Security By Design.
 \end{acknowledgments}

 \tableofcontents{}

 \chapter{Executive Summary}
 \label{cha:executive-summary}

\textbf{All the requirements for this project are statisfied specifically,}

\begin{description}
\item[Contents] \ \\
Our report has the following content:
\begin{enumerate}
 \item Chapter~\ref{cha:executive-summary}: Executive Summary
 \item Chapter~\ref{cha:exercise-4.6.3}: Exercise 4.6.3
   \begin{enumerate}
    \item Section~\ref{sec:problem-statement-1} Problem Statement
    \item Sectino~\ref{sec:relevant-code-1} Relevant Code
    \item Section~\ref{sec:test-cases} Test Cases
    \item Section~\ref{sec:exec-transcr} Execution Transcripts
    \item Section~\ref{sec:explanation-results} Explanation of Results
   \end{enumerate}
 \item Chapter~\ref{cha:exercise-4.6.4} Exercise 4.6.4
   \begin{enumerate}
    \item Section~\ref{sec:problem-statement-2} Problem Statement
    \item Section~\ref{sec:relevant-code-2} Relevant Code
    \item Section~\ref{sec:test-cases-2} Test Cases
    \item Section~\ref{sec:exec-transcr-2} Execution Transcripts
    \item Section~\ref{sec:explanation-results-2} Explanation of Results
    \end{enumerate}
 \item Chapter~\ref{cha:exercise-5.3.4} Exercise 5.3.4
   \begin{enumerate}
    \item Section~\ref{sec:problem-statement-3} Problem Statement
    \item Section~\ref{sec:relevant-code-3} Relevant Code
     \item Section~\ref{sec:test-cases-3} Test Cases
    \item Section~\ref{sec:exec-transcr-3} Execution Transcripts
    \item Section~\ref{sec:explanation-results-3} Explanation of Results
   \end{enumerate}
 \item Chapter~\ref{cha:exercise-5.3.5} Exercise 5.3.5
   \begin{enumerate}
    \item Section~\ref{sec:problem-statement-4} Problem Statement
    \item Section~\ref{sec:relevant-code-4} Relevant Code
     \item Section~\ref{sec:test-cases-4} Test Cases
    \item Section~\ref{sec:exec-transcr-4} Execution Transcripts
    \item Section~\ref{sec:explanation-results-4} Explanation of Results
   \end{enumerate}
 \item Chapter~\ref{cha:exercise-5.3.4} Exercise 6.2.1
   \begin{enumerate}
    \item Section~\ref{sec:problem-statement-5} Problem Statement
    \item Section~\ref{sec:relevant-code-5} Relevant Code
    \item Section~\ref{sec:exec-transcr-5} Execution Transcripts
    \item Section~\ref{sec:error-explanation} Explanation of Error for 6.2.1.4
   \end{enumerate}
\end{enumerate}
\item[Reproducibility in ML and \LaTeX{}] \ \\
 Our ML and \LaTeX{} source files compile with no errors.
\end{description}



%%------ Exercise 4.6.3 -------%%

 \chapter{Exercise 4.6.3}
 \label{cha:exercise-4.6.3}
  
 \section{Problem Statement}
 \label{sec:problem-statement-1}

  In this exercise we define five ML functions using \texttt{fun} and \texttt{val}.

\section{Relevant Code}
\label{sec:relevant-code-1}


 %%--- Problem A ---%%
 \subsection{4.6.3A}
 \label{sec:4.6.3a}

 In this we define a function that takes a 3-tuple of integers (x, y,
z) as input and returns the value corresponding to the sum x + y + z.


 %%--- Relevant Code ---%
  \lstset{frameround=tttt}
   \begin{lstlisting}[frame=tRBL]
    val funA1 = (fn (x,y,z) => x+y+z);
    fun funA2 (x,y,z) = x+y+z;
   \end{lstlisting}

 %%--- Problem B ---%%
 
 \subsection{4.6.3B}
 \label{sec:4.6.3b}

 In this we define a function that takes two integer inputs x and y
 (where x is supplied first followed by y) and returns the boolean
 value corresponding to x < y.

 %%--- Relevant Code ---%

   \lstset{frameround=tttt}
    \begin{lstlisting}[frame=tRBL]
     val funB1 = (fn x => (fn y => x < y));
     fun funB2 x y = x < y;
    \end{lstlisting}
 
 %%--- Problem C ---%%
 
 \subsection{4.6.3C}
 \label{sec:4.6.3c}

 In this we define a function that takes two strings s 1 and s 2 (where s 1 is supplied first followed by s 2 ) and concatentates them, where “ˆ” denotes string concatenation. For example, "Hi" ˆ " there" results in the string "Hi there".


 %%--- Relevant Code ---%

   \lstset{frameround=tttt}
    \begin{lstlisting}[frame=tRBL]
     val funC1 = (fn s1 => (fn s2 => s1 ^ s2));
     fun funC2 s1 s2 = s1 ^ s2;
    \end{lstlisting}

 %%--- Problem D ---%%

 \subsection{4.6.3D}
 \label{sec:4.6.3d}

 In this we define a function that takes two lists list 1 and list 2
 (where list 1 comes first) and appends them, where “@” denotes list
 append. For example [true,false] @ [false, false, false] results in
 the list [true,false,false,false,false].

 %%--- Relevant Code ---%

   \lstset{frameround=tttt}
    \begin{lstlisting}[frame=tRBL]
     val funD1 = (fn l1 => (fn l2 => l1@l2));
     fun funD2 l1 l2 = l1@l2; 
    \end{lstlisting}


 %%--- Problem E ---%%

 \subsection{4.6.3E}
 \label{sec:4.6.3e}

 In this we define a function that takes a pair of integers (x, y) and
 returns the larger of the two values. You note that the conditional
 statement if condition then a else b returns a if condition is true,
 other-wise it returns b.

 %%--- Relevant Code ---%

   \lstset{frameround=tttt}
    \begin{lstlisting}[frame=tRBL]
     val funE1 = (fn (x,y) => if (x>y) then x else y);
     fun funE2 (x,y) = if (x>y) then x else y;
    \end{lstlisting}

\section{Test Cases}
\label{sec:test-cases}

Below are the test cases to evaluate.
\begin{lstlisting}[frame = TB]
(**********)
(* Part A *)
(**********)

val testListA = [(1,2,3),(4,5,6),(7,8,9)]

val outputsA = map funA2 testListA

val testResultA = test463A funA1 funA2 testListA

(**********)
(* Part B *)
(**********)

val testListB = [(0,0),(1,2),(4,3)]

val outputsB = map (f2P funB1) testListB

val testResultB = test463B funB1 funB2 testListB

(**********)
(* Part C *)
(**********)

val testListC = [("Hi"," there!"),("Oh ","no!"),("What"," the ...")]

val outputsC = map (f2P funC1) testListC

val testResultC = test463B funC1 funC2 testListC


(**********)
(* Part D *)
(**********)

val testListD1 = [([0,1],[2,3,4]),([],[0,1])]
val testListD2 = [([true,true],[])]

val outputsD1 = map (f2P funD1) testListD1
val outputsD2 = map (f2P funD2) testListD2

val testResultD1 = test463B funD1 funD2 testListD1
val testResultD2 = test463B funD1 funD2 testListD2

(**********)
(* Part E *)
(**********)

val testListE = [(2,1),(5,5),(5,10)]

val sampleResultE = map funE1 testListE

val testResultE = test463A funE1 funE2 testListE

\end{lstlisting}


\section{Execution Transcripts}
\label{sec:exec-transcr}

\setcounter{sessioncount}{0}
\begin{session}
  \begin{scriptsize}
\begin{verbatim}

---------------------------------------------------------------------
       HOL-4 [Kananaskis 11 (stdknl, built Sat Aug 19 09:30:06 2017)]

       For introductory HOL help, type: help "hol";
       To exit type <Control>-D
---------------------------------------------------------------------
> > > > > > # # # # # # # # # val test463A = fn: ('a -> ''b) -> ('a -> ''b) -> 'a list -> bool
> > # # # # # # # # # # # val f2P = fn: ('a -> 'b -> 'c) -> 'a * 'b -> 'c
val test463B = fn:
   ('a -> 'b -> ''c) -> ('a -> 'b -> ''c) -> ('a * 'b) list -> bool
> 
*** Emacs/HOL command completed ***

> val funA1 = fn: int * int * int -> int
> val funA2 = fn: int * int * int -> int
> # # # # val outputsA = [6, 15, 24]: int list
val testListA = [(1, 2, 3), (4, 5, 6), (7, 8, 9)]: (int * int * int) list
val testResultA = true: bool
> val funB1 = fn: int -> int -> bool
> val funB2 = fn: int -> int -> bool
> > # # # # # val outputsB = [false, true, false]: bool list
val testListB = [(0, 0), (1, 2), (4, 3)]: (int * int) list
val testResultB = true: bool
> val funC1 = fn: string -> string -> string
> val funC2 = fn: string -> string -> string
> > # # # # val outputsC = ["Hi there!", "Oh no!", "What the ..."]: string list
val testListC = [("Hi", " there!"), ("Oh ", "no!"), ("What", " the ...")]:
   (string * string) list
val testResultC = true: bool
> val funD1 = fn: 'a list -> 'a list -> 'a list
> val funD2 = fn: 'a list -> 'a list -> 'a list
> # # # # # # # # val outputsD1 = [[0, 1, 2, 3, 4], [0, 1]]: int list list
val outputsD2 = [[true, true]]: bool list list
val testListD1 = [([0, 1], [2, 3, 4]), ([], [0, 1])]:
   (int list * int list) list
val testListD2 = [([true, true], [])]: (bool list * 'a list) list
val testResultD1 = true: bool
val testResultD2 = true: bool
> val funE1 = fn: int * int -> int
> val funE2 = fn: int * int -> int
> # # # # # val sampleResultE = [2, 5, 10]: int list
val testListE = [(2, 1), (5, 5), (5, 10)]: (int * int) list
val testResultE = true: bool
> 

\end{verbatim}
  \end{scriptsize}
\end{session}

\subsection{Explanation of Results}
\label{sec:explanation-results}
All the results in the test cases shows they are passed against the given test function.

%%---- Exercise 4.6.4 -----%

\chapter{Exercise 4.6.4}
\label{cha:exercise-4.6.4}

\section{Problem Statement}
\label{sec:problem-statement-2}
In this exercise we need to solve the list concatenation as stated below:

  In ML, define a function listSquares that when applied to the empty
  list of integers returns the empty list, and when applied to a
  non-empty list of integers returns a list where each element is
  squared. For example, listSquares [2,3,4] returns [4,9,16]. Define
  the function using a let expression in ML. A function that takes two
  lists list 1 and list 2 (where list 1 comes first) and appends them,
  where '@' denotes list append. For example [true,false] @ [false,
  false, false] results in the list [true,false,false,false,false].


\section{Relevant Code}
\label{sec:relevant-code-2}

\lstset{frameround=tttt}
\begin{lstlisting}[frame=tRBL]
 fun listSquares list =
let
  fun squareNum x = x*x
   in
    map squareNum list
end;
\end{lstlisting}

\section{Test Cases}
\label{sec:test-cases-2}

The required test cases are:
\begin{lstlisting}[frame = tRBL ]
val testList = [1,2,3,4,5]
\end{lstlisting}


\section{Execution Transcripts}
\label{sec:exec-transcr-2}

\setcounter{sessioncount}{0}
\begin{session}
  \begin{scriptsize}
\begin{verbatim}

---------------------------------------------------------------------
       HOL-4 [Kananaskis 11 (stdknl, built Sat Aug 19 09:30:06 2017)]

       For introductory HOL help, type: help "hol";
       To exit type <Control>-D
---------------------------------------------------------------------
> > > > # # # # # val listSquares = fn: int list -> int list
> val testList = [1, 2, 3, 4, 5]: int list
> val testResults = [1, 4, 9, 16, 25]: int list
> 
\end{verbatim}
  \end{scriptsize}
\end{session}

\subsection{Explanation of Results}
\label{sec:explanation-results-2}

The above transcript shows the given tests has been passed.


%%---- Exercise 5.3.4 -----%

\chapter{Exercise 5.3.4}
\label{cha:exercise-5.3.4}

\section{Problem Statement}
\label{sec:problem-statement-3}
In this exercise we need to define a function Filter in ML, whose
behavior is identical to filter. Note: you cannot use filter in the
definition of Filter. However, you can adapt the definition of filter
and use it in your definition.  Show test cases of your function
returning the expected results by comparing the outputs of both Filter
and filter.

\section{Relevant Code}
\label{sec:relevant-code-3}


\lstset{frameround=tttt}
\begin{lstlisting}[frame=tRBL]

fun Filter li list=
let

fun fnA li []=[]
  | fnA li xs=map li xs

fun fnB [] data=[]
  | fnB fail []=[]
  | fnB (b::bs) (x::xs)=if b then x::(fnB bs xs) else fnB bs xs
in
fnB (fnA li list) list
end;

\end{lstlisting}

\section{Test Cases}
\label{sec:test-cases-3}

The required test cases are:
\begin{lstlisting}[frame = tRBL ]
val testResults = Filter (fn x => x < 5) [1,2,3,4,5,6,7,8,9]
val testResults2 = Filter (fn x => x<5)[4,6]
\end{lstlisting}


\section{Execution Transcripts}
\label{sec:exec-transcr-3}

\setcounter{sessioncount}{0}
\begin{session}
  \begin{scriptsize}
\begin{verbatim}

---------------------------------------------------------------------
       HOL-4 [Kananaskis 11 (stdknl, built Sat Aug 19 09:30:06 2017)]

       For introductory HOL help, type: help "hol";
       To exit type <Control>-D
---------------------------------------------------------------------
> > > > # # # # # # # # # # # # val Filter = fn: ('a -> bool) -> 'a list -> 'a list
> > val testResults = [1, 2, 3, 4]: int list
> val testResults2 = [4]: int list
> 
\end{verbatim}
  \end{scriptsize}
\end{session}

\subsection{Explanation of Results}
\label{sec:explanation-results-3}

The above transcript shows the given tests has been passed.

%%---- Exercise 5.3.5 -----%

\chapter{Exercise 5.3.5}
\label{cha:exercise-5.3.5}

\section{Problem Statement}
\label{sec:problem-statement-4}
In this exercise we need to define a ML function addPairsGreaterThan n list, whose behavior is defined
as follows: (1) given an integer n, and (2) given a list of pairs of integers list, addPairsGreaterThan
n list will return a list of integers where each element is the sum of integer pairs in list where both
elements of the pairs are greater than n.

\section{Relevant Code}
\label{sec:relevant-code-4}


\lstset{frameround=tttt}
\begin{lstlisting}[frame=tRBL]

filter;

fun addPairsGreaterThan n list =
let
fun sumList [] = []
   |sumList ((x,y) :: xs) = (x+y) :: (sumList xs)

fun fil n (x,y) = (x>n andalso y>n)
in
sumList (filter (fil n) list)
end;


\end{lstlisting}

\section{Test Cases}
\label{sec:test-cases-4}

The required test cases are:
\begin{lstlisting}[frame = tRBL ]
addPairsGreaterThan 0 [(0,1),(2,0),(2,3),(4,5)];
\end{lstlisting}


\section{Execution Transcripts}
\label{sec:exec-transcr-4}

\setcounter{sessioncount}{0}
\begin{session}
  \begin{scriptsize}
\begin{verbatim}

---------------------------------------------------------------------
       HOL-4 [Kananaskis 11 (stdknl, built Sat Aug 19 09:30:06 2017)]

       For introductory HOL help, type: help "hol";
       To exit type <Control>-D
---------------------------------------------------------------------
> > > > val it = fn: ('a -> bool) -> 'a list -> 'a list
> # # # # # # # # val addPairsGreaterThan = fn: int -> (int * int) list -> int list
> > val it = [5, 9]: int list
> 
\end{verbatim}
  \end{scriptsize}
\end{session}

\subsection{Explanation of Results}
\label{sec:explanation-results-4}
The above transcript shows the given tests has been passed.

\chapter{Exercise 6.2.1}
\label{cha:exercise-6.2.1}

\section{Problem Statement}
\label{sec:problem-statement-5}

In this exercise we have to show the HOL equivalent code for the given sub-problems.

\section{Relevant Code}
\label{sec:relevant-code-5}

\subsection{6.2.1.1}
\label{sec:6.2.1.1}

 HOL equivalent of P(x) supset Q(y):

\lstset{frameround=tttt}
\begin{lstlisting}[frame=tRBL]
``P x ==> Q y``;
\end{lstlisting}

\subsection{6.2.1.2}
\label{sec:6.2.1.2}
P(x) supset Q(y) with x constrain to HOL type :num and y to Hol type :bool 

\lstset{frameround=tttt}
\begin{lstlisting}[frame=tRBL]
``(P:num -> bool) (x:num)  ==> (Q:bool->bool) (y:bool)``;
\end{lstlisting}

\subsection{6.2.1.3}
\label{sec:6.2.1.3}
forall x y P(x) supset Q(y) without specifying types

\lstset{frameround=tttt}
\begin{lstlisting}[frame=tRBL]
``!x y.(P x) ==> (Q y)``;
\end{lstlisting}

\subsection{6.2.1.4}
\label{sec:6.2.1.4}
forsome (x : num).R(x : α).

\lstset{frameround=tttt}
\begin{lstlisting}[frame=tRBL]
``?(x :num).(R (x :`a))``;
\end{lstlisting}

\subsection{6.2.1.5}
\label{sec:6.2.1.5}

\lstset{frameround=tttt}
\begin{lstlisting}[frame=tRBL]
``(~!x.(P x)\/(Q x))=(?x.(~(P x))/\~(Q x))``;
\end{lstlisting}

\subsection{6.2.1.6}
\label{sec:6.2.1.6}

All people are mortal, where P(x) represents x is a person and M(x) represents x is mortal.

\lstset{frameround=tttt}
\begin{lstlisting}[frame=tRBL]
``!x.(P x) ==> (M x)``;
\end{lstlisting}

\subsection{6.2.1.7}
\label{sec:6.2.1.7}

Some people are funny, where Funny(x) denotes x is funny.

\lstset{frameround=tttt}
\begin{lstlisting}[frame=tRBL]
``?x.(P x) ==> (Funny x)``;
\end{lstlisting}


\section{Execution Transcripts}
\label{sec:exec-transcr-5}

\setcounter{sessioncount}{0}
\begin{session}
  \begin{scriptsize}
\begin{verbatim}

---------------------------------------------------------------------
       HOL-4 [Kananaskis 11 (stdknl, built Sat Aug 19 09:30:06 2017)]

       For introductory HOL help, type: help "hol";
       To exit type <Control>-D
---------------------------------------------------------------------
> > > > # # # # # # # # # ** types trace now on
> # # # # # # # # # ** Unicode trace now off
> <<HOL message: inventing new type variable names: 'a, 'b>>
val it =
   ``(P :'a -> bool) (x :'a) ==> (Q :'b -> bool) (y :'b)``:
   term
> val it =
   ``(P :num -> bool) (x :num) ==> (Q :bool -> bool) (y :bool)``:
   term
> <<HOL message: inventing new type variable names: 'a, 'b>>
val it =
   ``!(x :'a) (y :'b). (P :'a -> bool) x ==> (Q :'b -> bool) y``:
   term
> <<HOL message: inventing new type variable names: 'a>>

Type inference failure: unable to infer a type for the application of

(x :num)

at line 22, character 16

to

(:` :'a)

on line 22, characters 18-22

unification failure message: Attempt to unify different type operators: num$num and min$fun
Exception-
   HOL_ERR
     {message =
      "on line 22, characters 18-22:\n\nType inference failure: unable to infer a type for the application of\n\n(x :num)
\n\nat line 22, character 16\n\nto\n\n(:` :'a)\n\non line 22, characters 18-22\n\nunification failure message:
 Attempt to unify different type operators: num$num and min$fun\n",
      origin_function = "type-analysis", origin_structure = "Preterm"} raised
> <<HOL message: inventing new type variable names: 'a>>
val it =
   ``~(!(x :'a). (P :'a -> bool) x \/ (Q :'a -> bool) x) <=>
  ?(x :'a). ~P x /\ ~Q x``:
   term
> <<HOL message: inventing new type variable names: 'a>>
val it =
   ``!(x :'a). (P :'a -> bool) x ==> (M :'a -> bool) x``:
   term
> <<HOL message: inventing new type variable names: 'a>>
val it =
   ``?(x :'a). (P :'a -> bool) x ==> (Funny :'a -> bool) x``:
   term
> 
 
\end{verbatim}
  \end{scriptsize}
\end{session}

\subsection{Explanation of Error for 6.2.1.4}
\label{sec:error-explanation}
 This cannot be evaluated, because x is specified to num then specify to alpha, So there is a type error

\chapter{Appendix A: Exercise 4.6.3}
\label{cha:appendix-a:-exercise463}

The following code is from the file ex-4-6-3Tests.sml.
\lstinputlisting{ML/ex-4-6-3Tests.sml}

\chapter{Appendix B: Exercise 4.6.4}
\label{cha:appendix-a:-exercise464}

The following code is from the file ex-4-6-4Tests.sml.
\lstinputlisting{ML/ex-4-6-4Tests.sml}

\chapter{Appendix C: Exercise 5.3.4}
\label{cha:appendix-a:-exercise534}

The following code is from the file ex-5-3-4Tests.sml.
\lstinputlisting{ML/ex-5-3-4Tests.sml}

\chapter{Appendix D: Exercise 5.3.5}
\label{cha:appendix-a:-exercise535}

The following code is from the file ex-5-3-5Tests.sml.
\lstinputlisting{ML/ex-5-3-5Tests.sml}

\chapter{Appendix E: Exercise 6.2.1}
\label{cha:appendix-e:-exercise621}

The following code is from the file ex-6-2-1.sml
\lstinputlisting{ML/ex-6-2-1.sml}


\end{document}



