\documentclass{report}
\title{Exam 2}
\author{Bharath Karumudi}
\date{\today}

%%........................................%%
%% Loading the Packages 
%%........................................%%
\usepackage{634format}
\usepackage{enumerate}
\usepackage{listings}
\usepackage{amsmath}
\usepackage{hyperref}
\usepackage{holtex}
\usepackage{holtexbasic}
\usepackage{amssymb}
% =====================================================================
%
% Macros for typesetting the HOL system manual
%
% =====================================================================

% ---------------------------------------------------------------------
% Abbreviations for words and phrases
% ---------------------------------------------------------------------

\newcommand\TUTORIAL{{\footnotesize\sl TUTORIAL}}
\newcommand\DESCRIPTION{{\footnotesize\sl DESCRIPTION}}
\newcommand\REFERENCE{{\footnotesize\sl REFERENCE}}
\newcommand\LOGIC{{\footnotesize\sl LOGIC}}
\newcommand\LIBRARIES{{\footnotesize\sl LIBRARIES}}

\newcommand{\bs}{\texttt{\char'134}} % backslash
\newcommand{\lb}{\texttt{\char'173}} % left brace
\newcommand{\rb}{\texttt{\char'175}} % right brace
\newcommand{\td}{\texttt{\char'176}} % tilde
\newcommand{\lt}{\texttt{\char'74}} % less than
\newcommand{\gt}{\texttt{\char'76}} % greater than
\newcommand{\dol}{\texttt{\char'44}} % dollar
% double back quotes ``
\newcommand{\dq}{\texttt{\char'140\char'140}}
%These macros were included by slind:

\newcommand{\holquote}[1]{\dq#1\dq}

\def\HOL{{\small HOL}}
\def\holn{\HOL}  % i.e. hol n(inety-eight), no digits in
                 % macro names is a bit of a pain; deciding to do away
                 % with hol98 nomenclature means that we just want to
                 % write HOL for hol98.
\def\holnversion{Kananaskis-7}
\def\holnsversion{Kananaskis~7} % version with space rather than hyphen
\def\LCF{{\small LCF}}
\def\LCFLSM{{\small LCF{\kern-.2em}{\normalsize\_}{\kern0.1em}LSM}}
\def\PPL{{\small PP}{\kern-.095em}$\lambda$}
\def\PPLAMBDA{{\small PPLAMBDA}}
\def\ML{{\small ML}}
\def\holmake{\texttt{Holmake}}

\newcommand\ie{\mbox{i{.}e{.}}}
\newcommand\eg{\mbox{e{.}g{.}}}
\newcommand\viz{\mbox{viz{.}}}
\newcommand\adhoc{\mbox{\it ad hoc}}
\newcommand\etal{{\it et al.\/}}
\newcommand\etc{\mbox{etc{.}}}

% ---------------------------------------------------------------------
% Simple abbreviations and macros for mathematical typesetting
% ---------------------------------------------------------------------

\newcommand\fun{{\to}}
\newcommand\prd{{\times}}

\newcommand\conj{\ \wedge\ }
\newcommand\disj{\ \vee\ }
\newcommand\imp{ \Rightarrow }
\newcommand\eqv{\ \equiv\ }
\newcommand\cond{\rightarrow}
\newcommand\vbar{\mid}
\newcommand\turn{\ \vdash\ }
\newcommand\hilbert{\varepsilon}
\newcommand\eqdef{\ \equiv\ }

\newcommand\natnums{\mbox{${\sf N}\!\!\!\!{\sf N}$}}
\newcommand\bools{\mbox{${\sf T}\!\!\!\!{\sf T}$}}

\newcommand\p{$\prime$}
\newcommand\f{$\forall$\ }
\newcommand\e{$\exists$\ }

\newcommand\orr{$\vee$\ }
\newcommand\negg{$\neg$\ }

\newcommand\arrr{$\rightarrow$}
\newcommand\hex{$\sharp $}

\newcommand{\uquant}[1]{\forall #1.\ }
\newcommand{\equant}[1]{\exists #1.\ }
\newcommand{\hquant}[1]{\hilbert #1.\ }
\newcommand{\iquant}[1]{\exists ! #1.\ }
\newcommand{\lquant}[1]{\lambda #1.\ }

\newcommand{\leave}[1]{\\[#1]\noindent}
\newcommand\entails{\mbox{\rule{.3mm}{4mm}\rule[2mm]{.2in}{.3mm}}}

% ---------------------------------------------------------------------
% Font-changing commands
% ---------------------------------------------------------------------

\newcommand{\theory}[1]{\hbox{{\small\tt #1}}}
\newcommand{\theoryimp}[1]{\texttt{#1}}

\newcommand{\con}[1]{{\sf #1}}
\newcommand{\rul}[1]{{\tt #1}}
\newcommand{\ty}[1]{\textsl{#1}}

\newcommand{\ml}[1]{\mbox{{\def\_{\char'137}\texttt{#1}}}}
\newcommand{\holtxt}[1]{\ml{#1}}
\newcommand\ms{\tt}
\newcommand{\s}[1]{{\small #1}}

\newcommand{\pin}[1]{{\bf #1}}
\def\m#1{\mbox{\normalsize$#1$}}

% ---------------------------------------------------------------------
% Abbreviations for particular mathematical constants etc.
% ---------------------------------------------------------------------

\newcommand\T{\con{T}}
\newcommand\F{\con{F}}
\newcommand\OneOne{\con{One\_One}}
\newcommand\OntoSubset{\con{Onto\_Subset}}
\newcommand\Onto{\con{Onto}}
\newcommand\TyDef{\con{Type\_Definition}}
\newcommand\Inv{\con{Inv}}
\newcommand\com{\con{o}}
\newcommand\Id{\con{I}}
\newcommand\MkPair{\con{Mk\_Pair}}
\newcommand\IsPair{\con{Is\_Pair}}
\newcommand\Fst{\con{Fst}}
\newcommand\Snd{\con{Snd}}
\newcommand\Suc{\con{Suc}}
\newcommand\Nil{\con{Nil}}
\newcommand\Cons{\con{Cons}}
\newcommand\Hd{\con{Hd}}
\newcommand\Tl{\con{Tl}}
\newcommand\Null{\con{Null}}
\newcommand\ListPrimRec{\con{List\_Prim\_Rec}}


\newcommand\SimpRec{\con{Simp\_Rec}}
\newcommand\SimpRecRel{\con{Simp\_Rec\_Rel}}
\newcommand\SimpRecFun{\con{Simp\_Rec\_Fun}}
\newcommand\PrimRec{\con{Prim\_Rec}}
\newcommand\PrimRecRel{\con{Prim\_Rec\_Rel}}
\newcommand\PrimRecFun{\con{Prim\_Rec\_Fun}}

\newcommand\bool{\ty{bool}}
\newcommand\num{\ty{num}}
\newcommand\ind{\ty{ind}}
\newcommand\lst{\ty{list}}

% ---------------------------------------------------------------------
% \minipagewidth = \textwidth minus 1.02 em
% ---------------------------------------------------------------------

\newlength{\minipagewidth}
\setlength{\minipagewidth}{\textwidth}
\addtolength{\minipagewidth}{-1.02em}

% ---------------------------------------------------------------------
% Environment for the items on the title page of a case study
% ---------------------------------------------------------------------

\newenvironment{inset}[1]{\noindent{\large\bf #1}\begin{list}%
{}{\setlength{\leftmargin}{\parindent}%
\setlength{\topsep}{-.1in}}\item }{\end{list}\vskip .4in}

% ---------------------------------------------------------------------
% Macros for little HOL sessions displayed in boxes.
%
% Usage: (1) \setcounter{sessioncount}{1} resets the session counter
%
%        (2) \begin{session}\begin{verbatim}
%             .
%              < lines from hol session >
%             .
%            \end{verbatim}\end{session}
%
%            typesets the session in a numbered box.
% ---------------------------------------------------------------------

\newlength{\hsbw}
\setlength{\hsbw}{\textwidth}
\addtolength{\hsbw}{-\arrayrulewidth}
\addtolength{\hsbw}{-\tabcolsep}
\newcommand\HOLSpacing{13pt}

\newcounter{sessioncount}
\setcounter{sessioncount}{0}

\newenvironment{session}{\begin{flushleft}
 \refstepcounter{sessioncount}
 \begin{tabular}{@{}|c@{}|@{}}\hline
 \begin{minipage}[b]{\hsbw}
 \vspace*{-.5pt}
 \begin{flushright}
 \rule{0.01in}{.15in}\rule{0.3in}{0.01in}\hspace{-0.35in}
 \raisebox{0.04in}{\makebox[0.3in][c]{\footnotesize\sl \thesessioncount}}
 \end{flushright}
 \vspace*{-.55in}
 \begingroup\small\baselineskip\HOLSpacing}{\endgroup\end{minipage}\\ \hline
 \end{tabular}
 \end{flushleft}}

% ---------------------------------------------------------------------
% Macro for boxed ML functions, etc.
%
% Usage: (1) \begin{holboxed}\begin{verbatim}
%               .
%               < lines giving names and types of mk functions >
%               .
%            \end{verbatim}\end{holboxed}
%
%            typesets the given lines in a box.
%
%            Conventions: lines are left-aligned under the "g" of begin,
%            and used to highlight primary reference for the ml function(s)
%            that appear in the box.
% ---------------------------------------------------------------------

\newenvironment{holboxed}{\begin{flushleft}
  \begin{tabular}{@{}|c@{}|@{}}\hline
  \begin{minipage}[b]{\hsbw}
% \vspace*{-.55in}
  \vspace*{.06in}
  \begingroup\small\baselineskip\HOLSpacing}{\endgroup\end{minipage}\\ \hline
  \end{tabular}
  \end{flushleft}}

% ---------------------------------------------------------------------
% Macro for unboxed ML functions, etc.
%
% Usage: (1) \begin{hol}\begin{verbatim}
%               .
%               < lines giving names and types of mk functions >
%               .
%            \end{verbatim}\end{hol}
%
%            typesets the given lines exactly like {boxed}, except there's
%            no box.
%
%            Conventions: lines are left-aligned under the "g" of begin,
%            and used to display ML code in verbatim, left aligned.
% ---------------------------------------------------------------------

\newenvironment{hol}{\begin{flushleft}
 \begin{tabular}{c@{}@{}}
 \begin{minipage}[b]{\hsbw}
% \vspace*{-.55in}
 \vspace*{.06in}
 \begingroup\small\baselineskip\HOLSpacing}{\endgroup\end{minipage}\\
 \end{tabular}
 \end{flushleft}}

% ---------------------------------------------------------------------
% Emphatic brackets
% ---------------------------------------------------------------------

\newcommand\leb{\lbrack\!\lbrack}
\newcommand\reb{\rbrack\!\rbrack}


% ---------------------------------------------------------------------
% Quotations
% ---------------------------------------------------------------------


%These macros were included by ap; they are used in Chapters 9 and 10
%of the HOL DESCRIPTION

\newcommand{\inds}%standard infinite set
 {\mbox{\rm I}}

\newcommand{\ch}%standard choice function
 {\mbox{\rm ch}}

\newcommand{\den}[1]%denotational brackets
 {[\![#1]\!]}

\newcommand{\two}%standard 2-element set
 {\mbox{\rm 2}}

\newcommand{\HOLquestionOneDate}{05 September 2019}
\newcommand{\HOLquestionOneTime}{18:56}
\begin{SaveVerbatim}{HOLquestionOneTheoremsquestionOneThm}
\HOLTokenTurnstile{} \HOLSymConst{\HOLTokenForall{}}\HOLBoundVar{signature}.
     \HOLConst{signVerify} (\HOLConst{pubK} \HOLFreeVar{TrueSignatures}) \HOLBoundVar{signature}
       (\HOLConst{SOME} \HOLStringLit{pubK GoodBooks}) \HOLSymConst{\HOLTokenEquiv{}}
     (\HOLBoundVar{signature} \HOLSymConst{=}
      \HOLConst{sign} (\HOLConst{privK} \HOLFreeVar{TrueSignatures})
        (\HOLConst{hash} (\HOLConst{SOME} \HOLStringLit{pubK GoodBooks})))
\end{SaveVerbatim}
\newcommand{\HOLquestionOneTheoremsquestionOneThm}{\UseVerbatim{HOLquestionOneTheoremsquestionOneThm}}
\newcommand{\HOLquestionOneTheorems}{
\HOLThmTag{question1}{question1Thm}\HOLquestionOneTheoremsquestionOneThm
}

\newcommand{\HOLquestionTwoDate}{03 September 2019}
\newcommand{\HOLquestionTwoTime}{21:32}
\begin{SaveVerbatim}{HOLquestionTwoDatatypescommands}
\HOLFreeVar{commands} = \HOLConst{pay} \HOLTokenBar{} \HOLConst{debit}
\end{SaveVerbatim}
\newcommand{\HOLquestionTwoDatatypescommands}{\UseVerbatim{HOLquestionTwoDatatypescommands}}
\begin{SaveVerbatim}{HOLquestionTwoDatatypeskeyPrinc}
\HOLFreeVar{keyPrinc} = \HOLConst{Staff} \HOLTyOp{people} \HOLTokenBar{} \HOLConst{Role} \HOLTyOp{roles} \HOLTokenBar{} \HOLConst{Ap} \HOLTyOp{num}
\end{SaveVerbatim}
\newcommand{\HOLquestionTwoDatatypeskeyPrinc}{\UseVerbatim{HOLquestionTwoDatatypeskeyPrinc}}
\begin{SaveVerbatim}{HOLquestionTwoDatatypespeople}
\HOLFreeVar{people} = \HOLConst{Alice} \HOLTokenBar{} \HOLConst{Bob}
\end{SaveVerbatim}
\newcommand{\HOLquestionTwoDatatypespeople}{\UseVerbatim{HOLquestionTwoDatatypespeople}}
\begin{SaveVerbatim}{HOLquestionTwoDatatypesprincipals}
\HOLFreeVar{principals} = \HOLConst{PR} \HOLTyOp{keyPrinc} \HOLTokenBar{} \HOLConst{Key} \HOLTyOp{keyPrinc}
\end{SaveVerbatim}
\newcommand{\HOLquestionTwoDatatypesprincipals}{\UseVerbatim{HOLquestionTwoDatatypesprincipals}}
\begin{SaveVerbatim}{HOLquestionTwoDatatypesroles}
\HOLFreeVar{roles} = \HOLConst{payer} \HOLTokenBar{} \HOLConst{payee}
\end{SaveVerbatim}
\newcommand{\HOLquestionTwoDatatypesroles}{\UseVerbatim{HOLquestionTwoDatatypesroles}}
\newcommand{\HOLquestionTwoDatatypes}{
\HOLquestionTwoDatatypescommands\HOLquestionTwoDatatypeskeyPrinc\HOLquestionTwoDatatypespeople\HOLquestionTwoDatatypesprincipals\HOLquestionTwoDatatypesroles}
\begin{SaveVerbatim}{HOLquestionTwoTheoremsquestionTwoThm}
\HOLTokenTurnstile{} (\HOLFreeVar{M}\HOLSymConst{,}\HOLFreeVar{Oi}\HOLSymConst{,}\HOLFreeVar{Os}) \HOLConst{sat} \HOLConst{Name} (\HOLConst{PR} (\HOLConst{Role} \HOLConst{payer})) \HOLConst{controls} \HOLConst{prop} \HOLConst{pay} \HOLSymConst{\HOLTokenImp{}}
   (\HOLFreeVar{M}\HOLSymConst{,}\HOLFreeVar{Oi}\HOLSymConst{,}\HOLFreeVar{Os}) \HOLConst{sat}
   \HOLConst{reps} (\HOLConst{Name} (\HOLConst{PR} (\HOLConst{Staff} \HOLConst{Alice}))) (\HOLConst{Name} (\HOLConst{PR} (\HOLConst{Role} \HOLConst{payer})))
     (\HOLConst{prop} \HOLConst{pay}) \HOLSymConst{\HOLTokenImp{}}
   (\HOLFreeVar{M}\HOLSymConst{,}\HOLFreeVar{Oi}\HOLSymConst{,}\HOLFreeVar{Os}) \HOLConst{sat}
   \HOLConst{Name} (\HOLConst{Key} (\HOLConst{Staff} \HOLConst{Alice})) \HOLConst{quoting} \HOLConst{Name} (\HOLConst{PR} (\HOLConst{Role} \HOLConst{payer})) \HOLConst{says}
   \HOLConst{prop} \HOLConst{pay} \HOLSymConst{\HOLTokenImp{}}
   (\HOLFreeVar{M}\HOLSymConst{,}\HOLFreeVar{Oi}\HOLSymConst{,}\HOLFreeVar{Os}) \HOLConst{sat} \HOLConst{prop} \HOLConst{pay} \HOLConst{impf} \HOLConst{prop} \HOLConst{debit} \HOLSymConst{\HOLTokenImp{}}
   (\HOLFreeVar{M}\HOLSymConst{,}\HOLFreeVar{Oi}\HOLSymConst{,}\HOLFreeVar{Os}) \HOLConst{sat}
   \HOLConst{Name} (\HOLConst{Key} (\HOLConst{Role} \HOLConst{payee})) \HOLConst{speaks_for} \HOLConst{Name} (\HOLConst{PR} (\HOLConst{Role} \HOLConst{payee})) \HOLSymConst{\HOLTokenImp{}}
   (\HOLFreeVar{M}\HOLSymConst{,}\HOLFreeVar{Oi}\HOLSymConst{,}\HOLFreeVar{Os}) \HOLConst{sat}
   \HOLConst{Name} (\HOLConst{Key} (\HOLConst{Role} \HOLConst{payee})) \HOLConst{says}
   \HOLConst{Name} (\HOLConst{Key} (\HOLConst{Staff} \HOLConst{Alice})) \HOLConst{speaks_for} \HOLConst{Name} (\HOLConst{PR} (\HOLConst{Staff} \HOLConst{Alice})) \HOLSymConst{\HOLTokenImp{}}
   (\HOLFreeVar{M}\HOLSymConst{,}\HOLFreeVar{Oi}\HOLSymConst{,}\HOLFreeVar{Os}) \HOLConst{sat}
   \HOLConst{Name} (\HOLConst{PR} (\HOLConst{Role} \HOLConst{payee})) \HOLConst{controls}
   \HOLConst{Name} (\HOLConst{Key} (\HOLConst{Staff} \HOLConst{Alice})) \HOLConst{speaks_for} \HOLConst{Name} (\HOLConst{PR} (\HOLConst{Staff} \HOLConst{Alice})) \HOLSymConst{\HOLTokenImp{}}
   (\HOLFreeVar{M}\HOLSymConst{,}\HOLFreeVar{Oi}\HOLSymConst{,}\HOLFreeVar{Os}) \HOLConst{sat}
   \HOLConst{Name} (\HOLConst{Key} (\HOLConst{Staff} \HOLConst{Bob})) \HOLConst{quoting} \HOLConst{Name} (\HOLConst{PR} (\HOLConst{Role} \HOLFreeVar{Operator})) \HOLConst{says}
   \HOLConst{prop} \HOLConst{debit}
\end{SaveVerbatim}
\newcommand{\HOLquestionTwoTheoremsquestionTwoThm}{\UseVerbatim{HOLquestionTwoTheoremsquestionTwoThm}}
\newcommand{\HOLquestionTwoTheorems}{
\HOLThmTag{question2}{question2Thm}\HOLquestionTwoTheoremsquestionTwoThm
}

%.........................................%%
%.........................................%%

\begin{document}
 \lstset{language=ML}
 \maketitle{}

 \begin{abstract}
   This project is to demonstrate the capabilities of implementing
   constructing and deconstructing HOL Terms using the tools and
   techniques - \LaTeX{}, AcuTeX, emacs and ML. 

   Each chapter documents the given problems with a structure of:
   \begin{enumerate}
   \item Problem Statement
   \item Relevant Code
   \item Execution Transcripts
   \item Explanation of results
   \end{enumerate}

 \end{abstract}


 \begin{acknowledgments}
  Professor Marvine Hamner and Professor Shiu-Kai Chin who taught the
  Certified Security By Design.
 \end{acknowledgments}

 \tableofcontents{}

 \chapter{Executive Summary}
 \label{cha:executive-summary}

\textbf{All requirements for this project are statisfied specifically,}
 and by using HOL solved the below questions:  

   \begin{enumerate}
   \item Question 1
   \item Question 2
   \item Question 3
   \end{enumerate}  



%%------Question 1 -------%%

 \chapter{Question 1}
 \label{cha:ques1}
  
 \section{Problem Statement}
 \label{sec:problem-statement-1}

 Sally purchases a new computer from a reputable company with the
 operating sys- tem and applications such as web browsers already
 installed.  Upon setting up her computer, Sally types the web address
 of her favorite Internet bookstore (Good- Books.com) into her web
 browser.  She connects with the bookstore’s web site and logs onto
 her account, which is handled by a secure portion of the web site
 that relies on a private and public key pair. Because she is a
 careful computer user, she verifies that she has connected to the
 real GoodBooks.com site by having her web browser authenticate the
 identity of the site. Her browser reports the following: The identity
 of this web site has been verified by TrueSignatures, Inc., a
 certificate authority you trust for this purpose.  Using her browser,
 she looks at the public-key certificate of GoodBooks.com and sees
 that it is signed by TrueSignatures, Inc.  She then makes her
 selections, places her order, enters her credit-card information, and
 then leaves the site.

\section{Relevant Code}
\label{sec:relevant-code-1}

\lstset{frameround=tttt}
\begin{lstlisting}[frame=tRBL]

 1. K TrueSignatures => TrueSignatures	 	        Trust assumption
 2. K TrueSignatures says K GoodBooks => GoodBooks	Public key Certificate
 3. TrueSignatures controls K GoodBooks => GoodBooks	Jurisdiction
 4. TrueSignatures says K GoodBooks => GoodBooks	1, 2 speaks for
 5. K GoodBooks	=> GoodBooks				3,4 controls

\end{lstlisting}

\lstinputlisting{../HOL/question1Script.sml}

\section{Execution Transcripts}
\label{sec:exec-transcr-1}

\setcounter{sessioncount}{0}
\begin{session}
  \begin{scriptsize}
\begin{verbatim}

---------------------------------------------------------------------
       HOL-4 [Kananaskis 11 (stdknl, built Sat Aug 19 09:30:06 2017)]

       For introductory HOL help, type: help "hol";
       To exit type <Control>-D
---------------------------------------------------------------------
> # # # # # # # # <<HOL message: inventing new type variable names: 'a>>
val question1Thm =
    []
|- !(signature :(string digest, 'a) asymMsg).
     signVerify (pubK (TrueSignatures :'a)) signature
       (SOME "pubK GoodBooks") <=>
     (signature =
      sign (privK TrueSignatures) (hash (SOME "pubK GoodBooks"))):
   thm
> > > # # # # # 
*** Time taken: 0.001s
 
\end{verbatim}
  \end{scriptsize}
\end{session}

\subsection{Explanation of Results}
\label{sec:explanation-results-1}
The above results shows that the requirements are satisfied.

%%------ Question 2 -------%%

 \chapter{Question 2}
 \label{cha:ques2}
  
 \section{Problem Statement}
 \label{sec:problem-statement-2}

 In this, we need to show the formal proof for Simple checking rule.
 Consider a situation where Alice wishes to purchase goods from Bob,
 her local grocer.  For simplicity, we assume that Alice and Bob are
 both depositors of the same bank. Alice wishes to pay Bob, but she
 does not have sufficient cash on hand (or she chooses not to use
 it). Instead, she writes a check to Bob, which is in fact an order to
 the bank to debit the given amount from her account and to pay it to
 Bob. Alice hands the check to Bob. Because Bob wishes to be paid, he
 heads to the bank, endorses Alice’s check by signing the back of it,
 and presents the check to the teller for payment.

\section{Relevant Code}
\label{sec:relevant-code-2}


\lstset{frameround=tttt}
\begin{lstlisting}[frame=tRBL]
Signature Bob | Signature Alice says  
                  (<pay amount, Bob> ^ <debit amt, acct Alice>)
Signature Alice => Alice
Signature Bob => Bob
<amt covered, acct Alice>
<amt covered, acct Alice> > Allice controls 
                  (<pay amount, Bob> ^ <debit amt, acct Alice>)
Bob reps Alice on (<pay amt, Bob> ^ <debit amt, acct Alice>)
   
\end{lstlisting}

\lstinputlisting{../HOL/question2Script.sml}


\section{Execution Transcripts}
\label{sec:exec-transcr-2}

\setcounter{sessioncount}{0}
\begin{session}
  \begin{scriptsize}
\begin{verbatim}

> # # # <<HOL message: Defined type: "commands">>
> # # # <<HOL message: Defined type: "people">>
> # # # <<HOL message: Defined type: "roles">>
> # # # <<HOL message: Defined type: "keyPrinc">>
> # # # <<HOL message: Defined type: "principals">>
> val question2Thm =
    []
|- ((M :(commands, 'b, principals, 'd, 'e) Kripke),(Oi :'d po),
    (Os :'e po)) sat
   Name (PR (Role payer)) controls
   (prop pay :(commands, principals, 'd, 'e) Form) ==>
   (M,Oi,Os) sat
   reps (Name (PR (Staff Alice))) (Name (PR (Role payer)))
     (prop pay :(commands, principals, 'd, 'e) Form) ==>
   (M,Oi,Os) sat
   Name (Key (Staff Alice)) quoting Name (PR (Role payer)) says
   (prop pay :(commands, principals, 'd, 'e) Form) ==>
   (M,Oi,Os) sat
   (prop pay :(commands, principals, 'd, 'e) Form) impf
   (prop debit :(commands, principals, 'd, 'e) Form) ==>
   (M,Oi,Os) sat
   ((Name (Key (Role payee)) speaks_for Name (PR (Role payee)))
      :(commands, principals, 'd, 'e) Form) ==>
   (M,Oi,Os) sat
   Name (Key (Role payee)) says
   ((Name (Key (Staff Alice)) speaks_for Name (PR (Staff Alice)))
      :(commands, principals, 'd, 'e) Form) ==>
   (M,Oi,Os) sat
   Name (PR (Role payee)) controls
   ((Name (Key (Staff Alice)) speaks_for Name (PR (Staff Alice)))
      :(commands, principals, 'd, 'e) Form) ==>
   (M,Oi,Os) sat
   Name (Key (Staff Bob)) quoting
   Name (PR (Role (Operator :roles))) says
   (prop debit :(commands, principals, 'd, 'e) Form):
   thm
val it = (): unit
> 
*** Emacs/HOL command completed ***

\end{verbatim}
  \end{scriptsize}
\end{session}

\subsection{Explanation of Results}
\label{sec:explanation-results-2}
The above results shows that the requirements are satisfied.


%%------ Question 3 -------%%

 \chapter{Question 3}
 \label{cha:exercise-11.6.3}
  
 \section{Problem Statement}
 \label{sec:problem-statement-3}

This question asks you to consider what you know about access control, particularly
discretionary access control as we’ve discussed, and delegation. As you consider this you
need to briefly state your opinion about whether or not delegation adds to the
“decidability” of access control, and most importantly if delegation adds to the security of
access control.

\section{Summary}
\label{sec:summary-3}

In the discretionary access control (DAC) model, every object (such as files and folders) has an owner, and the owner establishes access for the objects.
Many Operating systems such as Windows and most Unix-based systems, use the DAC model.
A common example of the DAC model is the NTFS used in Windows. 
NTFS provides security by allowing users and administrators to restrict the access to files and folders with permissions.

Microsoft systems identify users with SIDs and every object includes a discretionary access control list (DACL) that identifies who can access it in a system using the DAC model. 
THe DACL is a list of Access Control Entries. Each ACE is composed of a SID and the permissions granted to the SID.

If user create a file, they are designated as the owner and have explicit control over the file. 
As the owner, users can modify the permissions on the object by adding user or group accounts to the DACL and assigning the desired permissions. 
The DAC model is significantly more flexible than the Mandatory Access Control (MAC) model.

An inherent flaw associated with the DAC model is the susceptibility to Trojan horses. 
These Trojan horses injects the malware and when required they work on behalf of the user (delegation) unknowingly and make the damage to the system. 
Beside malware, if a user want to delegate the privileges, the delegation comes with trust and trustworthy. 
If the user is not trustworthy, then the delegation will create a loop-hole in the system security. 


\subsection{Conclusion}
\label{sec:conclusion-3}
In my opinion, delegation, especially with DAC, may potentially creates the damage to the system in long run.

\chapter{Appendix A: Question 1}
\label{cha:appendix-a:1}

The following code is from the file question1Script.sml
\lstinputlisting{../HOL/question1Script.sml}

\chapter{Appendix B: Question 2 }
\label{cha:appendix-b:2}

The following code is from the question2Script.sml
\lstinputlisting{../HOL/question2Script.sml}

\chapter{Appendix C}
\label{cha:appendix-c:3}
https://www.techotopia.com/index.php/Mandatory,_Discretionary,_Role_and_Rule_Based
_Access_Control

\end{document}