\documentclass{report}
\title{Exam 1}
\author{Bharath Karumudi}
\date{\today}

%%........................................%%
%% Loading the Packages 
%%........................................%%
\usepackage{634format}
\usepackage{enumerate}
\usepackage{listings}
\usepackage{amsmath}
\usepackage{hyperref}
\usepackage{holtex}
\usepackage{holtexbasic}
\usepackage{amssymb}
% =====================================================================
%
% Macros for typesetting the HOL system manual
%
% =====================================================================

% ---------------------------------------------------------------------
% Abbreviations for words and phrases
% ---------------------------------------------------------------------

\newcommand\TUTORIAL{{\footnotesize\sl TUTORIAL}}
\newcommand\DESCRIPTION{{\footnotesize\sl DESCRIPTION}}
\newcommand\REFERENCE{{\footnotesize\sl REFERENCE}}
\newcommand\LOGIC{{\footnotesize\sl LOGIC}}
\newcommand\LIBRARIES{{\footnotesize\sl LIBRARIES}}

\newcommand{\bs}{\texttt{\char'134}} % backslash
\newcommand{\lb}{\texttt{\char'173}} % left brace
\newcommand{\rb}{\texttt{\char'175}} % right brace
\newcommand{\td}{\texttt{\char'176}} % tilde
\newcommand{\lt}{\texttt{\char'74}} % less than
\newcommand{\gt}{\texttt{\char'76}} % greater than
\newcommand{\dol}{\texttt{\char'44}} % dollar
% double back quotes ``
\newcommand{\dq}{\texttt{\char'140\char'140}}
%These macros were included by slind:

\newcommand{\holquote}[1]{\dq#1\dq}

\def\HOL{{\small HOL}}
\def\holn{\HOL}  % i.e. hol n(inety-eight), no digits in
                 % macro names is a bit of a pain; deciding to do away
                 % with hol98 nomenclature means that we just want to
                 % write HOL for hol98.
\def\holnversion{Kananaskis-7}
\def\holnsversion{Kananaskis~7} % version with space rather than hyphen
\def\LCF{{\small LCF}}
\def\LCFLSM{{\small LCF{\kern-.2em}{\normalsize\_}{\kern0.1em}LSM}}
\def\PPL{{\small PP}{\kern-.095em}$\lambda$}
\def\PPLAMBDA{{\small PPLAMBDA}}
\def\ML{{\small ML}}
\def\holmake{\texttt{Holmake}}

\newcommand\ie{\mbox{i{.}e{.}}}
\newcommand\eg{\mbox{e{.}g{.}}}
\newcommand\viz{\mbox{viz{.}}}
\newcommand\adhoc{\mbox{\it ad hoc}}
\newcommand\etal{{\it et al.\/}}
\newcommand\etc{\mbox{etc{.}}}

% ---------------------------------------------------------------------
% Simple abbreviations and macros for mathematical typesetting
% ---------------------------------------------------------------------

\newcommand\fun{{\to}}
\newcommand\prd{{\times}}

\newcommand\conj{\ \wedge\ }
\newcommand\disj{\ \vee\ }
\newcommand\imp{ \Rightarrow }
\newcommand\eqv{\ \equiv\ }
\newcommand\cond{\rightarrow}
\newcommand\vbar{\mid}
\newcommand\turn{\ \vdash\ }
\newcommand\hilbert{\varepsilon}
\newcommand\eqdef{\ \equiv\ }

\newcommand\natnums{\mbox{${\sf N}\!\!\!\!{\sf N}$}}
\newcommand\bools{\mbox{${\sf T}\!\!\!\!{\sf T}$}}

\newcommand\p{$\prime$}
\newcommand\f{$\forall$\ }
\newcommand\e{$\exists$\ }

\newcommand\orr{$\vee$\ }
\newcommand\negg{$\neg$\ }

\newcommand\arrr{$\rightarrow$}
\newcommand\hex{$\sharp $}

\newcommand{\uquant}[1]{\forall #1.\ }
\newcommand{\equant}[1]{\exists #1.\ }
\newcommand{\hquant}[1]{\hilbert #1.\ }
\newcommand{\iquant}[1]{\exists ! #1.\ }
\newcommand{\lquant}[1]{\lambda #1.\ }

\newcommand{\leave}[1]{\\[#1]\noindent}
\newcommand\entails{\mbox{\rule{.3mm}{4mm}\rule[2mm]{.2in}{.3mm}}}

% ---------------------------------------------------------------------
% Font-changing commands
% ---------------------------------------------------------------------

\newcommand{\theory}[1]{\hbox{{\small\tt #1}}}
\newcommand{\theoryimp}[1]{\texttt{#1}}

\newcommand{\con}[1]{{\sf #1}}
\newcommand{\rul}[1]{{\tt #1}}
\newcommand{\ty}[1]{\textsl{#1}}

\newcommand{\ml}[1]{\mbox{{\def\_{\char'137}\texttt{#1}}}}
\newcommand{\holtxt}[1]{\ml{#1}}
\newcommand\ms{\tt}
\newcommand{\s}[1]{{\small #1}}

\newcommand{\pin}[1]{{\bf #1}}
\def\m#1{\mbox{\normalsize$#1$}}

% ---------------------------------------------------------------------
% Abbreviations for particular mathematical constants etc.
% ---------------------------------------------------------------------

\newcommand\T{\con{T}}
\newcommand\F{\con{F}}
\newcommand\OneOne{\con{One\_One}}
\newcommand\OntoSubset{\con{Onto\_Subset}}
\newcommand\Onto{\con{Onto}}
\newcommand\TyDef{\con{Type\_Definition}}
\newcommand\Inv{\con{Inv}}
\newcommand\com{\con{o}}
\newcommand\Id{\con{I}}
\newcommand\MkPair{\con{Mk\_Pair}}
\newcommand\IsPair{\con{Is\_Pair}}
\newcommand\Fst{\con{Fst}}
\newcommand\Snd{\con{Snd}}
\newcommand\Suc{\con{Suc}}
\newcommand\Nil{\con{Nil}}
\newcommand\Cons{\con{Cons}}
\newcommand\Hd{\con{Hd}}
\newcommand\Tl{\con{Tl}}
\newcommand\Null{\con{Null}}
\newcommand\ListPrimRec{\con{List\_Prim\_Rec}}


\newcommand\SimpRec{\con{Simp\_Rec}}
\newcommand\SimpRecRel{\con{Simp\_Rec\_Rel}}
\newcommand\SimpRecFun{\con{Simp\_Rec\_Fun}}
\newcommand\PrimRec{\con{Prim\_Rec}}
\newcommand\PrimRecRel{\con{Prim\_Rec\_Rel}}
\newcommand\PrimRecFun{\con{Prim\_Rec\_Fun}}

\newcommand\bool{\ty{bool}}
\newcommand\num{\ty{num}}
\newcommand\ind{\ty{ind}}
\newcommand\lst{\ty{list}}

% ---------------------------------------------------------------------
% \minipagewidth = \textwidth minus 1.02 em
% ---------------------------------------------------------------------

\newlength{\minipagewidth}
\setlength{\minipagewidth}{\textwidth}
\addtolength{\minipagewidth}{-1.02em}

% ---------------------------------------------------------------------
% Environment for the items on the title page of a case study
% ---------------------------------------------------------------------

\newenvironment{inset}[1]{\noindent{\large\bf #1}\begin{list}%
{}{\setlength{\leftmargin}{\parindent}%
\setlength{\topsep}{-.1in}}\item }{\end{list}\vskip .4in}

% ---------------------------------------------------------------------
% Macros for little HOL sessions displayed in boxes.
%
% Usage: (1) \setcounter{sessioncount}{1} resets the session counter
%
%        (2) \begin{session}\begin{verbatim}
%             .
%              < lines from hol session >
%             .
%            \end{verbatim}\end{session}
%
%            typesets the session in a numbered box.
% ---------------------------------------------------------------------

\newlength{\hsbw}
\setlength{\hsbw}{\textwidth}
\addtolength{\hsbw}{-\arrayrulewidth}
\addtolength{\hsbw}{-\tabcolsep}
\newcommand\HOLSpacing{13pt}

\newcounter{sessioncount}
\setcounter{sessioncount}{0}

\newenvironment{session}{\begin{flushleft}
 \refstepcounter{sessioncount}
 \begin{tabular}{@{}|c@{}|@{}}\hline
 \begin{minipage}[b]{\hsbw}
 \vspace*{-.5pt}
 \begin{flushright}
 \rule{0.01in}{.15in}\rule{0.3in}{0.01in}\hspace{-0.35in}
 \raisebox{0.04in}{\makebox[0.3in][c]{\footnotesize\sl \thesessioncount}}
 \end{flushright}
 \vspace*{-.55in}
 \begingroup\small\baselineskip\HOLSpacing}{\endgroup\end{minipage}\\ \hline
 \end{tabular}
 \end{flushleft}}

% ---------------------------------------------------------------------
% Macro for boxed ML functions, etc.
%
% Usage: (1) \begin{holboxed}\begin{verbatim}
%               .
%               < lines giving names and types of mk functions >
%               .
%            \end{verbatim}\end{holboxed}
%
%            typesets the given lines in a box.
%
%            Conventions: lines are left-aligned under the "g" of begin,
%            and used to highlight primary reference for the ml function(s)
%            that appear in the box.
% ---------------------------------------------------------------------

\newenvironment{holboxed}{\begin{flushleft}
  \begin{tabular}{@{}|c@{}|@{}}\hline
  \begin{minipage}[b]{\hsbw}
% \vspace*{-.55in}
  \vspace*{.06in}
  \begingroup\small\baselineskip\HOLSpacing}{\endgroup\end{minipage}\\ \hline
  \end{tabular}
  \end{flushleft}}

% ---------------------------------------------------------------------
% Macro for unboxed ML functions, etc.
%
% Usage: (1) \begin{hol}\begin{verbatim}
%               .
%               < lines giving names and types of mk functions >
%               .
%            \end{verbatim}\end{hol}
%
%            typesets the given lines exactly like {boxed}, except there's
%            no box.
%
%            Conventions: lines are left-aligned under the "g" of begin,
%            and used to display ML code in verbatim, left aligned.
% ---------------------------------------------------------------------

\newenvironment{hol}{\begin{flushleft}
 \begin{tabular}{c@{}@{}}
 \begin{minipage}[b]{\hsbw}
% \vspace*{-.55in}
 \vspace*{.06in}
 \begingroup\small\baselineskip\HOLSpacing}{\endgroup\end{minipage}\\
 \end{tabular}
 \end{flushleft}}

% ---------------------------------------------------------------------
% Emphatic brackets
% ---------------------------------------------------------------------

\newcommand\leb{\lbrack\!\lbrack}
\newcommand\reb{\rbrack\!\rbrack}


% ---------------------------------------------------------------------
% Quotations
% ---------------------------------------------------------------------


%These macros were included by ap; they are used in Chapters 9 and 10
%of the HOL DESCRIPTION

\newcommand{\inds}%standard infinite set
 {\mbox{\rm I}}

\newcommand{\ch}%standard choice function
 {\mbox{\rm ch}}

\newcommand{\den}[1]%denotational brackets
 {[\![#1]\!]}

\newcommand{\two}%standard 2-element set
 {\mbox{\rm 2}}

\newcommand{\HOLquestionOneDate}{05 September 2019}
\newcommand{\HOLquestionOneTime}{18:56}
\begin{SaveVerbatim}{HOLquestionOneTheoremsquestionOneThm}
\HOLTokenTurnstile{} \HOLSymConst{\HOLTokenForall{}}\HOLBoundVar{signature}.
     \HOLConst{signVerify} (\HOLConst{pubK} \HOLFreeVar{TrueSignatures}) \HOLBoundVar{signature}
       (\HOLConst{SOME} \HOLStringLit{pubK GoodBooks}) \HOLSymConst{\HOLTokenEquiv{}}
     (\HOLBoundVar{signature} \HOLSymConst{=}
      \HOLConst{sign} (\HOLConst{privK} \HOLFreeVar{TrueSignatures})
        (\HOLConst{hash} (\HOLConst{SOME} \HOLStringLit{pubK GoodBooks})))
\end{SaveVerbatim}
\newcommand{\HOLquestionOneTheoremsquestionOneThm}{\UseVerbatim{HOLquestionOneTheoremsquestionOneThm}}
\newcommand{\HOLquestionOneTheorems}{
\HOLThmTag{question1}{question1Thm}\HOLquestionOneTheoremsquestionOneThm
}

\newcommand{\HOLquestionTwoDate}{03 September 2019}
\newcommand{\HOLquestionTwoTime}{21:32}
\begin{SaveVerbatim}{HOLquestionTwoDatatypescommands}
\HOLFreeVar{commands} = \HOLConst{pay} \HOLTokenBar{} \HOLConst{debit}
\end{SaveVerbatim}
\newcommand{\HOLquestionTwoDatatypescommands}{\UseVerbatim{HOLquestionTwoDatatypescommands}}
\begin{SaveVerbatim}{HOLquestionTwoDatatypeskeyPrinc}
\HOLFreeVar{keyPrinc} = \HOLConst{Staff} \HOLTyOp{people} \HOLTokenBar{} \HOLConst{Role} \HOLTyOp{roles} \HOLTokenBar{} \HOLConst{Ap} \HOLTyOp{num}
\end{SaveVerbatim}
\newcommand{\HOLquestionTwoDatatypeskeyPrinc}{\UseVerbatim{HOLquestionTwoDatatypeskeyPrinc}}
\begin{SaveVerbatim}{HOLquestionTwoDatatypespeople}
\HOLFreeVar{people} = \HOLConst{Alice} \HOLTokenBar{} \HOLConst{Bob}
\end{SaveVerbatim}
\newcommand{\HOLquestionTwoDatatypespeople}{\UseVerbatim{HOLquestionTwoDatatypespeople}}
\begin{SaveVerbatim}{HOLquestionTwoDatatypesprincipals}
\HOLFreeVar{principals} = \HOLConst{PR} \HOLTyOp{keyPrinc} \HOLTokenBar{} \HOLConst{Key} \HOLTyOp{keyPrinc}
\end{SaveVerbatim}
\newcommand{\HOLquestionTwoDatatypesprincipals}{\UseVerbatim{HOLquestionTwoDatatypesprincipals}}
\begin{SaveVerbatim}{HOLquestionTwoDatatypesroles}
\HOLFreeVar{roles} = \HOLConst{payer} \HOLTokenBar{} \HOLConst{payee}
\end{SaveVerbatim}
\newcommand{\HOLquestionTwoDatatypesroles}{\UseVerbatim{HOLquestionTwoDatatypesroles}}
\newcommand{\HOLquestionTwoDatatypes}{
\HOLquestionTwoDatatypescommands\HOLquestionTwoDatatypeskeyPrinc\HOLquestionTwoDatatypespeople\HOLquestionTwoDatatypesprincipals\HOLquestionTwoDatatypesroles}
\begin{SaveVerbatim}{HOLquestionTwoTheoremsquestionTwoThm}
\HOLTokenTurnstile{} (\HOLFreeVar{M}\HOLSymConst{,}\HOLFreeVar{Oi}\HOLSymConst{,}\HOLFreeVar{Os}) \HOLConst{sat} \HOLConst{Name} (\HOLConst{PR} (\HOLConst{Role} \HOLConst{payer})) \HOLConst{controls} \HOLConst{prop} \HOLConst{pay} \HOLSymConst{\HOLTokenImp{}}
   (\HOLFreeVar{M}\HOLSymConst{,}\HOLFreeVar{Oi}\HOLSymConst{,}\HOLFreeVar{Os}) \HOLConst{sat}
   \HOLConst{reps} (\HOLConst{Name} (\HOLConst{PR} (\HOLConst{Staff} \HOLConst{Alice}))) (\HOLConst{Name} (\HOLConst{PR} (\HOLConst{Role} \HOLConst{payer})))
     (\HOLConst{prop} \HOLConst{pay}) \HOLSymConst{\HOLTokenImp{}}
   (\HOLFreeVar{M}\HOLSymConst{,}\HOLFreeVar{Oi}\HOLSymConst{,}\HOLFreeVar{Os}) \HOLConst{sat}
   \HOLConst{Name} (\HOLConst{Key} (\HOLConst{Staff} \HOLConst{Alice})) \HOLConst{quoting} \HOLConst{Name} (\HOLConst{PR} (\HOLConst{Role} \HOLConst{payer})) \HOLConst{says}
   \HOLConst{prop} \HOLConst{pay} \HOLSymConst{\HOLTokenImp{}}
   (\HOLFreeVar{M}\HOLSymConst{,}\HOLFreeVar{Oi}\HOLSymConst{,}\HOLFreeVar{Os}) \HOLConst{sat} \HOLConst{prop} \HOLConst{pay} \HOLConst{impf} \HOLConst{prop} \HOLConst{debit} \HOLSymConst{\HOLTokenImp{}}
   (\HOLFreeVar{M}\HOLSymConst{,}\HOLFreeVar{Oi}\HOLSymConst{,}\HOLFreeVar{Os}) \HOLConst{sat}
   \HOLConst{Name} (\HOLConst{Key} (\HOLConst{Role} \HOLConst{payee})) \HOLConst{speaks_for} \HOLConst{Name} (\HOLConst{PR} (\HOLConst{Role} \HOLConst{payee})) \HOLSymConst{\HOLTokenImp{}}
   (\HOLFreeVar{M}\HOLSymConst{,}\HOLFreeVar{Oi}\HOLSymConst{,}\HOLFreeVar{Os}) \HOLConst{sat}
   \HOLConst{Name} (\HOLConst{Key} (\HOLConst{Role} \HOLConst{payee})) \HOLConst{says}
   \HOLConst{Name} (\HOLConst{Key} (\HOLConst{Staff} \HOLConst{Alice})) \HOLConst{speaks_for} \HOLConst{Name} (\HOLConst{PR} (\HOLConst{Staff} \HOLConst{Alice})) \HOLSymConst{\HOLTokenImp{}}
   (\HOLFreeVar{M}\HOLSymConst{,}\HOLFreeVar{Oi}\HOLSymConst{,}\HOLFreeVar{Os}) \HOLConst{sat}
   \HOLConst{Name} (\HOLConst{PR} (\HOLConst{Role} \HOLConst{payee})) \HOLConst{controls}
   \HOLConst{Name} (\HOLConst{Key} (\HOLConst{Staff} \HOLConst{Alice})) \HOLConst{speaks_for} \HOLConst{Name} (\HOLConst{PR} (\HOLConst{Staff} \HOLConst{Alice})) \HOLSymConst{\HOLTokenImp{}}
   (\HOLFreeVar{M}\HOLSymConst{,}\HOLFreeVar{Oi}\HOLSymConst{,}\HOLFreeVar{Os}) \HOLConst{sat}
   \HOLConst{Name} (\HOLConst{Key} (\HOLConst{Staff} \HOLConst{Bob})) \HOLConst{quoting} \HOLConst{Name} (\HOLConst{PR} (\HOLConst{Role} \HOLFreeVar{Operator})) \HOLConst{says}
   \HOLConst{prop} \HOLConst{debit}
\end{SaveVerbatim}
\newcommand{\HOLquestionTwoTheoremsquestionTwoThm}{\UseVerbatim{HOLquestionTwoTheoremsquestionTwoThm}}
\newcommand{\HOLquestionTwoTheorems}{
\HOLThmTag{question2}{question2Thm}\HOLquestionTwoTheoremsquestionTwoThm
}

%.........................................%%
%.........................................%%

\begin{document}
 \lstset{language=ML}
 \maketitle{}

 \begin{abstract}
   This project is to demonstrate the capabilities of implementing
   constructing and deconstructing HOL Terms using the tools and
   techniques - \LaTeX{}, AcuTeX, emacs and ML. 

   Each chapter documents the given problems with a structure of:
   \begin{enumerate}
   \item Problem Statement
   \item Relevant Code
   \item Execution Transcripts
   \item Explanation of results
   \end{enumerate}

 \end{abstract}


 \begin{acknowledgments}
  Professor Marvine Hamner and Professor Shiu-Kai Chin who taught the
  Certified Security By Design.
 \end{acknowledgments}

 \tableofcontents{}

 \chapter{Executive Summary}
 \label{cha:executive-summary}

\textbf{All requirements for this project are statisfied specifically,}
 and by using HOL solved the below questions:  

   \begin{enumerate}
   \item Question 1
   \item Question 2
   \item Question 3
   \end{enumerate}  



%%------Question 1 -------%%

 \chapter{Question 1}
 \label{cha:ques1}
  
 \section{Problem Statement}
 \label{sec:problem-statement-1}

Consider the following situation. Jack is taking an Amtrak train to visit his parents over the
winter break from school, University of the Brilliant. He purchased his ticket online and has
printed it out.
Jack arrives at the train depot and boards the train. As the train pulls out of the depot the
conductor approaches and asks for Jack’s ticket. Using the programming and mathematical
languages you have learned in CIS634 (emacs, LaTex, HOL, etc.) generate a report for Amtrak’s
Security \& Operations manager giving a formal proof for why Jack should be allowed to
complete his trip.

\section{Relevant Code}
\label{sec:relevant-code-1}

\lstset{frameround=tttt}
\begin{lstlisting}[frame=tRBL]
  1. Jack says <travel, train>                        Jack's Request  
  2. Amtrack controls (Jack controls <travel, train>) Access Policy  
  3. ticket => Amtrack                                Trust Assumption  
  4. ticket says (Jack controls <travel, train>)      Jack's Ticket  
  5. <travel, train>                                  1,2,3,4 ticket rule  
\end{lstlisting}

\lstinputlisting{../HOL/question1Script.sml}

\section{Execution Transcripts}
\label{sec:exec-transcr-1}

\setcounter{sessioncount}{0}
\begin{session}
  \begin{scriptsize}
\begin{verbatim}


---------------------------------------------------------------------
       HOL-4 [Kananaskis 11 (stdknl, built Sat Aug 19 09:30:06 2017)]

       For introductory HOL help, type: help "hol";
       To exit type <Control>-D
---------------------------------------------------------------------
[extending loadPath with Holmakefile INCLUDES variable]
> > > > # # # # # # # # # ** types trace now on
> *** Globals.show_assums now true ***
> # # # # # # # # # ** Unicode trace now off
> # # <<HOL message: Defined type: "commands">>
> # # # <<HOL message: Defined type: "staff">>
> val term1 =
   ``Name Jack says (prop travel :(commands, staff, 'd, 'e) Form)``:
   term
> val term2 =
   ``Name Jack controls (prop travel :(commands, staff, 'd, 'e) Form)``:
   term
> val term3 =
   ``(prop travel :(commands, staff, 'd, 'e) Form) impf
  (prop deny :(commands, staff, 'd, 'e) Form)``:
   term
> val term4 =
   ``((Name (Ticket :staff) speaks_for Name Amtrack)
     :(commands, staff, 'd, 'e) Form)``:
   term
> 
*** Emacs/HOL command completed ***

> # # # # # # # # # # val question1Thm =
   
[((M :(commands, 'b, staff, 'd, 'e) Kripke),(Oi :'d po),(Os :'e po)) sat
 Name Jack controls (prop travel :(commands, staff, 'd, 'e) Form),
 ((M :(commands, 'b, staff, 'd, 'e) Kripke),(Oi :'d po),(Os :'e po)) sat
 (prop travel :(commands, staff, 'd, 'e) Form) impf
 (prop deny :(commands, staff, 'd, 'e) Form),
 ((M :(commands, 'b, staff, 'd, 'e) Kripke),(Oi :'d po),(Os :'e po)) sat
 Name Jack says (prop travel :(commands, staff, 'd, 'e) Form)]
|- ((M :(commands, 'b, staff, 'd, 'e) Kripke),(Oi :'d po),
    (Os :'e po)) sat
   Name Jack says (prop deny :(commands, staff, 'd, 'e) Form):
   thm
> 

 
\end{verbatim}
  \end{scriptsize}
\end{session}

\subsection{Explanation of Results}
\label{sec:explanation-results-1}
The above results shows that the requirements are satisfied.

%%------ Question 2 -------%%

 \chapter{Question 2}
 \label{cha:ques2}
  
 \section{Problem Statement}
 \label{sec:problem-statement-2}

Consider the following situation. Alice has a lock box at her bank, the Bank of Riches. To
access her lock box Alice needs an official state-issued identification card, typically her driver’s
license, DL\#2507, and the key to her lock box,\#113.
Alice wants to store a valuable document in her lock box. She proceeds to her bank and
presents her credentials in order to access her lock box. Using the programming and
mathematical languages you have learned in CIS634 (emacs, LaTex, HOL, etc.) generate a report
for the Bank’s manager giving a formal proof for why Alice should have access. Hint: the key
will act as a “ticket” while the driver’s license can be checked against an access control list
(ACL).

\section{Relevant Code}
\label{sec:relevant-code-2}


\lstset{frameround=tttt}
\begin{lstlisting}[frame=tRBL]
  1. Face says <enter, lockerroom>                            Request
  2. Bank controls ((Alice controls <113,lockerroom>)         Jurisdiction
                    Alice controls <enter, lockerroom>
  3. Bank says ((Alice controls <113, lockerroom>)            Access Policy 
                Alice controls <enter, lockerroom>
  4. DMV controls (Face => Alice)                             Jurisdiction
  5. License says (Face => Alice)                             Drivers License
  6. License => DMV                                           Trust Assumption
  7. Bank controls (Alice controls <113, lockerroom>)         Jurisdiction
  8. key says (Alice controls <113, lockerroom>)              Ticket
  9. key => Bank                                              Trust Assumption
  10.<enter, lockerroom>                                      1,2,3,4,5,6,7,8,9
                                                              ID \& Ticket Rule    
\end{lstlisting}

\lstinputlisting{../HOL/question1Script.sml}


\section{Execution Transcripts}
\label{sec:exec-transcr-2}

\setcounter{sessioncount}{0}
\begin{session}
  \begin{scriptsize}
\begin{verbatim}

---------------------------------------------------------------------
       HOL-4 [Kananaskis 11 (stdknl, built Sat Aug 19 09:30:06 2017)]

       For introductory HOL help, type: help "hol";
       To exit type <Control>-D
---------------------------------------------------------------------
[extending loadPath with Holmakefile INCLUDES variable]
> > > > # # # # # # # # # ** types trace now on
> *** Globals.show_assums now true ***
> # # # # # # # # # ** Unicode trace now off
> 
> 
> # # <<HOL message: Defined type: "commands">>
> # # # <<HOL message: Defined type: "people">>
> # # <<HOL message: Defined type: "roles">>
> # # <<HOL message: Defined type: "keyPrinc">>
> # # <<HOL message: Defined type: "principals">>
> val question2Thm =
    []
|- ((M :(commands, 'b, principals, 'd, 'e) Kripke),(Oi :'d po),
    (Os :'e po)) sat
   Name (PR (Role Commander)) controls
   (prop go :(commands, principals, 'd, 'e) Form) ==>
   (M,Oi,Os) sat
   reps (Name (PR (Staff Alice))) (Name (PR (Role Commander)))
     (prop go :(commands, principals, 'd, 'e) Form) ==>
   (M,Oi,Os) sat
   Name (Key (Staff Alice)) quoting Name (PR (Role Commander)) says
   (prop go :(commands, principals, 'd, 'e) Form) ==>
   (M,Oi,Os) sat
   (prop go :(commands, principals, 'd, 'e) Form) impf
   (prop nogo :(commands, principals, 'd, 'e) Form) ==>
   (M,Oi,Os) sat
   ((Name (Key (Role CA)) speaks_for Name (PR (Role CA)))
      :(commands, principals, 'd, 'e) Form) ==>
   (M,Oi,Os) sat
   Name (Key (Role CA)) says
   ((Name (Key (Staff Alice)) speaks_for Name (PR (Staff Alice)))
      :(commands, principals, 'd, 'e) Form) ==>
   (M,Oi,Os) sat
   Name (PR (Role CA)) controls
   ((Name (Key (Staff Alice)) speaks_for Name (PR (Staff Alice)))
      :(commands, principals, 'd, 'e) Form) ==>
   (M,Oi,Os) sat
   Name (Key (Staff Bank)) quoting
   Name (PR (Role (Operator :roles))) says
   (prop nogo :(commands, principals, 'd, 'e) Form):
   thm
val it = (): unit
> 
*** Emacs/HOL command completed ***

\end{verbatim}
  \end{scriptsize}
\end{session}

\subsection{Explanation of Results}
\label{sec:explanation-results-2}
The above results shows that the requirements are satisfied.


%%------ Question 3 -------%%

 \chapter{Question 3}
 \label{cha:exercise-11.6.3}
  
 \section{Problem Statement}
 \label{sec:problem-statement-3}

Discretionary access control poses real-world challenges in security. This problem is 
intended to help you explore what those challenges are and some potential ways to mitigate
negative impacts due to these challenges. Consider the 2005 paper, “On Safety in Discretionary
Access Control,” written by Li and Tripunitara at Purdue University. I’ve uploaded this paper to
our course website for you. You can use this paper, your textbook, or other resources at your
disposal. But remember, whatever you use you will be on the clock to get your exam done.
Given what you now know, this question asks you to consider whether or not you believe
that safety in existing discretionary access control schemes being “decidable” means that these
schemes are in fact safe. Or, if this merely means that we can decide what the level of safety in
these schemes is. You are to include a 1 page summary in your report that expresses your
thinking about this topic including a sentence that summarizes your conclusion.

\section{Summary}
\label{sec:summary-3}

As mentioned by Ninghui et. al in the research paper “On safety in Discretionary Access
Control” Safety analysis decides whether rights can be leaked to unauthorized principals in
future states. Safety analysis was shown to be undecidable in the HRU scheme. Safety analysis,
first formulated by Harrison, et. al for the access matrix model has been recognized as
fundamental problem in access control. Solworth and Sloan safety is undecidable in DAC and
use this assertion as the motivation for introducing a new access control scheme based on
labels and relabeling that has decidable safety properties.
DAC cannot be equated to the HRU scheme for the following reasons:
First, the HRU scheme can be used to encode schemes that are not DAC schemes; therefore,
the fact that safety is undecidable in the HRU scheme should not lead one to conclude that
safety is undecidable in DAC.
Second, features in DAC cannot always be encoded in the HRU scheme. For example, some DAC
schemes require that each object be owned by exactly one subject; thus removal of a subject
who has the ownership of some objects requires the transfer of ownership to some other
subject (often times the owner of the subject being removed) so that this property is
maintained. Both the removal of the subject and the transfer of ownership of objects it owns
occur in a single state-change.
The algorithm by authors, suggests that safety in these DAC schemes can be efficiently decided
and there is no need to invent new access control schemes with decidable safety as the primary
goal. The authors show that the proposed implementation of DAC schemes in the Solworth-
Sloan scheme has significant deficiencies. Two particular limitations that discussed are the lack
of support for removing subjects and objects and the inability to ensure that an object has only
one owner, as required by DAC schemes. Solworth and Sloan presented a new DAC scheme
based on labels and relabeling rules. To our knowledge, the work by Solworth and Sloan was
the first to directly address safety in DAC.
There is considerable overhead in implementing a relatively simple DAC scheme (SDCO) in the
Solworth-Sloan scheme. For each object, we need to create a set of labels whose size is linear in
the number of the subjects in the state. We also need to create a set of tags whose size is
exponential in the number rights in the system. These tags are used to define groups, and the
therefore, the number of entries in all the sets of patterns is also exponential in the number of
rights in the system. This is considerable overhead considering the simplicity of SDCO, and the
fact that one can “directly” implement it, with efficiently decidable safety.
Conclusion, the existing general DAC schemes is decidable and there is no need to invent new
DAC schemes with decidable safety as the primary goal.


\subsection{Conclusion}
\label{sec:conclusion-3}
The existing general DAC schemes is decidable and there is no need to invent new
DAC schemes with decidable safety as the primary goal.


\chapter{Appendix A: Question 1}
\label{cha:appendix-a:chapter13}

The following code is from the file question1Script.sml
\lstinputlisting{../HOL/question1Script.sml}

\chapter{Appendix B: Question 2 }
\label{cha:appendix-a:chapter13b}

The following code is from the question2Script.sml
\lstinputlisting{../HOL/question2Script.sml}

\chapter{Appendix C}
\label{cha:appendix-a:chapter13b}
Ninghui Li , Mahesh V. Tripunitara, On Safety in Discretionary Access Control, Proceedings of the 2005 IEEE Symposium on Security and Privacy.

\end{document}