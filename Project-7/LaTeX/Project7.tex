\documentclass{report}
\title{Project 7}
\author{Bharath Karumudi}
\date{\today}

%%........................................%%
%% Loading the Packages 
%%........................................%%
\usepackage{634format}
\usepackage{enumerate}
\usepackage{listings}
\usepackage{amsmath}
\usepackage{hyperref}
\usepackage{holtex}
\usepackage{holtexbasic}
\usepackage{amssymb}
\input{commands}
\input{../HOL/HOLReports/HOLcipher}
\input{../HOL/HOLReports/HOLcryptoExercises}
%.........................................%%
%.........................................%%

\begin{document}
 \lstset{language=ML}
 \maketitle{}

 \begin{abstract}
   This project is to demonstrate the capabilities of implementing
   constructing and deconstructing HOL Terms using the tools and
   techniques - \LaTeX{}, AcuTeX, emacs and ML. 

   Each chapter documents the given problems with a structure of:
   \begin{enumerate}
   \item Problem Statement
   \item Relevant Code
   \item Execution Transcripts
   \item Explanation of results
   \end{enumerate}

 \end{abstract}


 \begin{acknowledgments}
  Professor Marvine Hamner and Professor Shiu-Kai Chin who taught the
  Certified Security By Design.
 \end{acknowledgments}

 \tableofcontents{}

 \chapter{Executive Summary}
 \label{cha:executive-summary}

\textbf{All requirements for this project are statisfied specifically,}
 and by using HOL proved the below theorems:
\begin{quote}



\end{quote}


%%------ Exercise 15.6.1 -------%%

 \chapter{Exercise 15.6.1}
 \label{cha:exercise-15.6.1}
  
 \section{Problem Statement}
 \label{sec:problem-statement-1}
In this we need to prove:  

\HOLTokenTurnstile{} \HOLSymConst{\HOLTokenForall{}}\HOLBoundVar{keyAlice} \HOLBoundVar{k} \HOLBoundVar{text}.
     (\HOLConst{deciphS} \HOLBoundVar{keyAlice} (\HOLConst{Es} \HOLBoundVar{k} (\HOLConst{SOME} \HOLBoundVar{text})) \HOLSymConst{=}
      \HOLConst{SOME} \HOLStringLit{This is from Alice}) \HOLSymConst{\HOLTokenEquiv{}}
     (\HOLBoundVar{k} \HOLSymConst{=} \HOLBoundVar{keyAlice}) \HOLSymConst{\HOLTokenConj{}} (\HOLBoundVar{text} \HOLSymConst{=} \HOLStringLit{This is from Alice})

\HOLTokenTurnstile{} \HOLSymConst{\HOLTokenForall{}}\HOLBoundVar{P} \HOLBoundVar{message}.
     (\HOLConst{deciphP} (\HOLConst{pubK} \HOLBoundVar{P}) \HOLFreeVar{enMsg} \HOLSymConst{=} \HOLConst{SOME} \HOLBoundVar{message}) \HOLSymConst{\HOLTokenEquiv{}}
     (\HOLFreeVar{enMsg} \HOLSymConst{=} \HOLConst{Ea} (\HOLConst{privK} \HOLBoundVar{P}) (\HOLConst{SOME} \HOLBoundVar{message}))

\section{Proof of exercise15_6_1a_thm}
\label{sec:proof-1a}

\subsection{Relevant Code}
\label{sec:relevant-code-1a}
\lstinputlisting{HOLCode/ex-15-6-1a.sml}

\subsection{Execution Transcripts}
\label{sec:exec-transcr-1a}

\setcounter{sessioncount}{0}
\begin{session}
  \begin{scriptsize}
\begin{verbatim}

---------------------------------------------------------------------
       HOL-4 [Kananaskis 11 (stdknl, built Sat Aug 19 09:30:06 2017)]

       For introductory HOL help, type: help "hol";
       To exit type <Control>-D
---------------------------------------------------------------------
[extending loadPath with Holmakefile INCLUDES variable]
> > > > 

val exercise15_6_1a_thm =
    []
|- !(key :symKey) (enMsg :'a symMsg) (message :'a).
     (deciphS key enMsg = SOME message) <=>
     (enMsg = Es key (SOME message)):
   thm
> 
 
\end{verbatim}
  \end{scriptsize}
\end{session}

\subsection{Explanation of Results}
\label{sec:explanation-results-1}
The above results shows that the requirements are satisfied.

\section{Proof of exercise15_6_1b_thm}
\label{sec:proof-1b}

\subsection{Relevant Code}
\label{sec:relevant-code-1b}
\lstinputlisting{HOLCode/ex-15-6-1b.sml}


\subsection{Execution Transcripts}
\label{sec:exec-transcr-1b}

\setcounter{sessioncount}{0}
\begin{session}
  \begin{scriptsize}
\begin{verbatim}

---------------------------------------------------------------------
       HOL-4 [Kananaskis 11 (stdknl, built Sat Aug 19 09:30:06 2017)]

       For introductory HOL help, type: help "hol";
       To exit type <Control>-D
---------------------------------------------------------------------
[extending loadPath with Holmakefile INCLUDES variable]
> > > > 

> # # # # # # # # # Meson search level: ........................
val exercise15_6_1b_thm =
    []
|- !(keyAlice :symKey) (k :symKey) (text :string).
     (deciphS keyAlice (Es k (SOME text)) =
      SOME "This is from Alice") <=>
     (k = keyAlice) /\ (text = "This is from Alice"):
   thm
>  
\end{verbatim}
  \end{scriptsize}
\end{session}

\subsection{Explanation of Results}
\label{sec:explanation-results-1b}
The above results shows that the requirements are satisfied.


%%------ Exercise 15.6.2 -------%%


 \chapter{Exercise 15.6.2}
 \label{cha:exercise-15.6.2}
  
 \section{Problem Statement}
 \label{sec:problem-statement-2}
In this we need to prove:  

\HOLTokenTurnstile{} \HOLSymConst{\HOLTokenForall{}}\HOLBoundVar{key} \HOLBoundVar{text}.
     (\HOLConst{deciphP} (\HOLConst{pubK} \HOLFreeVar{Alice}) (\HOLConst{Ea} \HOLBoundVar{key} (\HOLConst{SOME} \HOLBoundVar{text})) \HOLSymConst{=}
      \HOLConst{SOME} \HOLStringLit{This is from Alice}) \HOLSymConst{\HOLTokenEquiv{}}
     (\HOLBoundVar{key} \HOLSymConst{=} \HOLConst{privK} \HOLFreeVar{Alice}) \HOLSymConst{\HOLTokenConj{}} (\HOLBoundVar{text} \HOLSymConst{=} \HOLStringLit{This is from Alice})


\HOLTokenTurnstile{} \HOLSymConst{\HOLTokenForall{}}\HOLBoundVar{key} \HOLBoundVar{text}.
     (\HOLConst{deciphP} (\HOLConst{pubK} \HOLFreeVar{Alice}) (\HOLConst{Ea} \HOLBoundVar{key} (\HOLConst{SOME} \HOLBoundVar{text})) \HOLSymConst{=}
      \HOLConst{SOME} \HOLStringLit{This is from Alice}) \HOLSymConst{\HOLTokenEquiv{}}
     (\HOLBoundVar{key} \HOLSymConst{=} \HOLConst{privK} \HOLFreeVar{Alice}) \HOLSymConst{\HOLTokenConj{}} (\HOLBoundVar{text} \HOLSymConst{=} \HOLStringLit{This is from Alice})

\section{Proof of exercise15_6_2a_thm}
\label{sec:proof-2a}

\subsection{Relevant Code}
\label{sec:relevant-code-2a}
\lstinputlisting{HOLCode/ex-15-6-2a.sml}

\subsection{Execution Transcripts}
\label{sec:exec-transcr-2a}

\setcounter{sessioncount}{0}
\begin{session}
  \begin{scriptsize}
\begin{verbatim}

---------------------------------------------------------------------
       HOL-4 [Kananaskis 11 (stdknl, built Sat Aug 19 09:30:06 2017)]

       For introductory HOL help, type: help "hol";
       To exit type <Control>-D
---------------------------------------------------------------------
> # # # # # # # <<HOL message: inventing new type variable names: 'a, 'b>>
Meson search level: ..........
val exercise15_6_2a_thm =
    []
|- !(P :'a) (message :'b).
     (deciphP (pubK P) (enMsg :('b, 'a) asymMsg) = SOME message) <=>
     (enMsg = Ea (privK P) (SOME message)):
   thm
> 
 
\end{verbatim}
  \end{scriptsize}
\end{session}

\subsection{Explanation of Results}
\label{sec:explanation-results-2a}
The above results shows that the requirements are satisfied.


\section{Proof of exercise15_6_2b_thm}
\label{sec:proof-2b}

\subsection{Relevant Code}
\label{sec:relevant-code-2b}
\lstinputlisting{HOLCode/ex-15-6-2b.sml}

\subsection{Execution Transcripts}
\label{sec:exec-transcr-2b}

\setcounter{sessioncount}{0}
\begin{session}
  \begin{scriptsize}
\begin{verbatim}

---------------------------------------------------------------------
       HOL-4 [Kananaskis 11 (stdknl, built Sat Aug 19 09:30:06 2017)]

       For introductory HOL help, type: help "hol";
       To exit type <Control>-D
---------------------------------------------------------------------
> # # # # # # # # <<HOL message: inventing new type variable names: 'a>>
val exercise15_6_2b_thm =
    []
|- !(key :'a pKey) (text :string).
     (deciphP (pubK (Alice :'a)) (Ea key (SOME text)) =
      SOME "This is from Alice") <=>
     (key = privK Alice) /\ (text = "This is from Alice"):
   thm
> 

\end{verbatim}
  \end{scriptsize}
\end{session}

\subsection{Explanation of Results}
\label{sec:explanation-results-2b}
The above results shows that the requirements are satisfied.



%%------ Exercise 15.6.3 -------%%


 \chapter{Exercise 15.6.3}
 \label{cha:exercise-15.6.3}
  
 \section{Problem Statement}
 \label{sec:problem-statement-3}
In this we need to prove:  

\HOLTokenTurnstile{} \HOLSymConst{\HOLTokenForall{}}\HOLBoundVar{signature}.
     \HOLConst{signVerify} (\HOLConst{pubK} \HOLFreeVar{Alice}) \HOLBoundVar{signature}
       (\HOLConst{SOME} \HOLStringLit{This is from Alice}) \HOLSymConst{\HOLTokenEquiv{}}
     (\HOLBoundVar{signature} \HOLSymConst{=}
      \HOLConst{sign} (\HOLConst{privK} \HOLFreeVar{Alice}) (\HOLConst{hash} (\HOLConst{SOME} \HOLStringLit{This is from Alice})))

\subsection{Relevant Code}
\label{sec:relevant-code-1a}
\lstinputlisting{HOLCode/ex-15-6-3.sml}


\subsection{Execution Transcripts}
\label{sec:exec-transcr-2a}

\setcounter{sessioncount}{0}
\begin{session}
  \begin{scriptsize}
\begin{verbatim}

---------------------------------------------------------------------
       HOL-4 [Kananaskis 11 (stdknl, built Sat Aug 19 09:30:06 2017)]

       For introductory HOL help, type: help "hol";
       To exit type <Control>-D
---------------------------------------------------------------------
> # # # # # # # <<HOL message: inventing new type variable names: 'a>>
val exercise15_6_3_thm =
    []
|- !(signature :(string digest, 'a) asymMsg).
     signVerify (pubK (Alice :'a)) signature
       (SOME "This is from Alice") <=>
     (signature = sign (privK Alice) (hash (SOME "This is from Alice"))):
   thm
> 
 
\end{verbatim}
  \end{scriptsize}
\end{session}


\subsection{Explanation of Results}
\label{sec:explanation-results-1}
The above results shows that the requirements are satisfied.


%%------ Appendix -------%%

\chapter{Appendix A: cipherScript}
\label{cha:appendix-a:chapter15}

The following code is from the file cipherScript.sml
\lstinputlisting{../HOL/cipherScript.sml}

\chapter{Appendix B: cryptoExericisesScript }
\label{cha:appendix-a:chapter15}

The following code is from the file cryptoExercisesScript.sml
\lstinputlisting{../HOL/cryptoExercisesScript.sml}


\end{document}