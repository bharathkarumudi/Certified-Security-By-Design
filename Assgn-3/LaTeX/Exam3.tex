\documentclass{report}
\title{Exam 3}
\author{Bharath Karumudi}
\date{\today}

%%........................................%%
%% Loading the Packages 
%%........................................%%
\usepackage{634format}
\usepackage{enumerate}
\usepackage{listings}
\usepackage{amsmath}
\usepackage{hyperref}
\usepackage{holtex}
\usepackage{holtexbasic}
\usepackage{amssymb}
\input{commands}
\newcommand{\HOLquestionTwoDate}{15 September 2019}
\newcommand{\HOLquestionTwoTime}{13:22}
\begin{SaveVerbatim}{HOLquestionTwoDatatypescommands}
\HOLFreeVar{commands} = \HOLConst{grant} \HOLTokenBar{} \HOLConst{deny}
\end{SaveVerbatim}
\newcommand{\HOLquestionTwoDatatypescommands}{\UseVerbatim{HOLquestionTwoDatatypescommands}}
\begin{SaveVerbatim}{HOLquestionTwoDatatypeskeyPrinc}
\HOLFreeVar{keyPrinc} = \HOLConst{Staff} \HOLTyOp{people} \HOLTokenBar{} \HOLConst{Role} \HOLTyOp{roles} \HOLTokenBar{} \HOLConst{Ap} \HOLTyOp{num}
\end{SaveVerbatim}
\newcommand{\HOLquestionTwoDatatypeskeyPrinc}{\UseVerbatim{HOLquestionTwoDatatypeskeyPrinc}}
\begin{SaveVerbatim}{HOLquestionTwoDatatypespeople}
\HOLFreeVar{people} = \HOLConst{Eve} \HOLTokenBar{} \HOLConst{Bob}
\end{SaveVerbatim}
\newcommand{\HOLquestionTwoDatatypespeople}{\UseVerbatim{HOLquestionTwoDatatypespeople}}
\begin{SaveVerbatim}{HOLquestionTwoDatatypesprincipals}
\HOLFreeVar{principals} = \HOLConst{PR} \HOLTyOp{keyPrinc} \HOLTokenBar{} \HOLConst{Key} \HOLTyOp{keyPrinc}
\end{SaveVerbatim}
\newcommand{\HOLquestionTwoDatatypesprincipals}{\UseVerbatim{HOLquestionTwoDatatypesprincipals}}
\begin{SaveVerbatim}{HOLquestionTwoDatatypesroles}
\HOLFreeVar{roles} = \HOLConst{owner} \HOLTokenBar{} \HOLConst{requester}
\end{SaveVerbatim}
\newcommand{\HOLquestionTwoDatatypesroles}{\UseVerbatim{HOLquestionTwoDatatypesroles}}
\newcommand{\HOLquestionTwoDatatypes}{
\HOLquestionTwoDatatypescommands\HOLquestionTwoDatatypeskeyPrinc\HOLquestionTwoDatatypespeople\HOLquestionTwoDatatypesprincipals\HOLquestionTwoDatatypesroles}
\begin{SaveVerbatim}{HOLquestionTwoTheoremsquestionTwoThm}
\HOLTokenTurnstile{} (\HOLFreeVar{M}\HOLSymConst{,}\HOLFreeVar{Oi}\HOLSymConst{,}\HOLFreeVar{Os}) \HOLConst{sat} \HOLConst{Name} (\HOLConst{PR} (\HOLConst{Role} \HOLConst{owner})) \HOLConst{controls} \HOLConst{prop} \HOLConst{grant} \HOLSymConst{\HOLTokenImp{}}
   (\HOLFreeVar{M}\HOLSymConst{,}\HOLFreeVar{Oi}\HOLSymConst{,}\HOLFreeVar{Os}) \HOLConst{sat}
   \HOLConst{reps} (\HOLConst{Name} (\HOLConst{PR} (\HOLConst{Staff} \HOLConst{Eve}))) (\HOLConst{Name} (\HOLConst{PR} (\HOLConst{Role} \HOLConst{owner})))
     (\HOLConst{prop} \HOLConst{grant}) \HOLSymConst{\HOLTokenImp{}}
   (\HOLFreeVar{M}\HOLSymConst{,}\HOLFreeVar{Oi}\HOLSymConst{,}\HOLFreeVar{Os}) \HOLConst{sat}
   \HOLConst{Name} (\HOLConst{Key} (\HOLConst{Staff} \HOLConst{Eve})) \HOLConst{quoting} \HOLConst{Name} (\HOLConst{PR} (\HOLConst{Role} \HOLConst{owner})) \HOLConst{says}
   \HOLConst{prop} \HOLConst{grant} \HOLSymConst{\HOLTokenImp{}}
   (\HOLFreeVar{M}\HOLSymConst{,}\HOLFreeVar{Oi}\HOLSymConst{,}\HOLFreeVar{Os}) \HOLConst{sat} \HOLConst{prop} \HOLConst{grant} \HOLConst{impf} \HOLConst{prop} \HOLConst{deny} \HOLSymConst{\HOLTokenImp{}}
   (\HOLFreeVar{M}\HOLSymConst{,}\HOLFreeVar{Oi}\HOLSymConst{,}\HOLFreeVar{Os}) \HOLConst{sat}
   \HOLConst{Name} (\HOLConst{Key} (\HOLConst{Role} \HOLConst{requester})) \HOLConst{speaks_for}
   \HOLConst{Name} (\HOLConst{PR} (\HOLConst{Role} \HOLConst{requester})) \HOLSymConst{\HOLTokenImp{}}
   (\HOLFreeVar{M}\HOLSymConst{,}\HOLFreeVar{Oi}\HOLSymConst{,}\HOLFreeVar{Os}) \HOLConst{sat}
   \HOLConst{Name} (\HOLConst{Key} (\HOLConst{Role} \HOLConst{requester})) \HOLConst{says}
   \HOLConst{Name} (\HOLConst{Key} (\HOLConst{Staff} \HOLConst{Eve})) \HOLConst{speaks_for} \HOLConst{Name} (\HOLConst{PR} (\HOLConst{Staff} \HOLConst{Eve})) \HOLSymConst{\HOLTokenImp{}}
   (\HOLFreeVar{M}\HOLSymConst{,}\HOLFreeVar{Oi}\HOLSymConst{,}\HOLFreeVar{Os}) \HOLConst{sat}
   \HOLConst{Name} (\HOLConst{PR} (\HOLConst{Role} \HOLConst{requester})) \HOLConst{controls}
   \HOLConst{Name} (\HOLConst{Key} (\HOLConst{Staff} \HOLConst{Eve})) \HOLConst{speaks_for} \HOLConst{Name} (\HOLConst{PR} (\HOLConst{Staff} \HOLConst{Eve})) \HOLSymConst{\HOLTokenImp{}}
   (\HOLFreeVar{M}\HOLSymConst{,}\HOLFreeVar{Oi}\HOLSymConst{,}\HOLFreeVar{Os}) \HOLConst{sat}
   \HOLConst{Name} (\HOLConst{Key} (\HOLConst{Staff} \HOLConst{Bob})) \HOLConst{quoting} \HOLConst{Name} (\HOLConst{PR} (\HOLConst{Role} \HOLFreeVar{Operator})) \HOLConst{says}
   \HOLConst{prop} \HOLConst{deny}
\end{SaveVerbatim}
\newcommand{\HOLquestionTwoTheoremsquestionTwoThm}{\UseVerbatim{HOLquestionTwoTheoremsquestionTwoThm}}
\newcommand{\HOLquestionTwoTheorems}{
\HOLThmTag{question2}{question2Thm}\HOLquestionTwoTheoremsquestionTwoThm
}

%.........................................%%
%.........................................%%

\begin{document}
 \lstset{language=ML}
 \maketitle{}

 \begin{abstract}
   This project is to demonstrate the capabilities of implementing
   constructing and deconstructing HOL Terms using the tools and
   techniques - \LaTeX{}, AcuTeX, emacs and ML. 

   Each chapter documents the given problems with a structure of:
   \begin{enumerate}
   \item Problem Statement
   \item Relevant Code
   \item Execution Transcripts
   \item Explanation of results
   \end{enumerate}

 \end{abstract}


 \begin{acknowledgments}
  Professor Marvine Hamner and Professor Shiu-Kai Chin who taught the
  Certified Security By Design.
 \end{acknowledgments}

 \tableofcontents{}

 \chapter{Executive Summary}
 \label{cha:executive-summary}

\textbf{All requirements for this project are statisfied specifically,}
 and by using HOL solved the below questions: 

   \begin{enumerate}
   \item Question 1
   \item Question 2
   \item Question 3
   \end{enumerate}  



%%------Question 1 -------%%

 \chapter{Question 1}
 \label{cha:ques1}
  
 \section{Problem Statement}
 \label{sec:problem-statement-1}

Create Access control matrix by considering the following relationships:

File 1 comprises the engineering files, including the engineering
drawings Alice is working on for a new, custom, design which are
considered proprietary.  File 2 are the accounting department’s
records including for Mr. Coyote’s account with Acme, past billing,
payments, project data (hours) for custom projects, etc., as well as
various company data regarding assets, liabilities, revenue, etc.
File 3 is a relational database that stores all Acme’s basic account
information for its customers, e.g. Customer name, address, type of
business, contacts with the customer (data, Acme rep contacting
customer, product(s) discussed, type of contact e.g. phone call,
letter, etc.), and past purchases including date of purchase,
product(s) purchased, etc. This relational database is managed by
Acme’s Customer Relationship Department.  File 4 contains Acme’s CEO,
Mr. Knowsitall, personal files. For example, this include the
correspondence with Acme’s customers as is typically generated by
accessing Acme’s relational database to merge product data, accounting
data used to create proposed product pricing, etc.

\section{Relevant Code}
\label{sec:relevant-code-1}

\begin{table}[h!]
\centering
\begin{tabular}{lllll}
 & File1 & File2 & File3 & File4 \\
Alice & r,w,x & r,w &  &  \\
Bob & r & r,w,x & r &  \\
Eve & r & r,w & r &  \\
Knowsitall & r,x & r,x & r,w,x & r,w,x
\end{tabular}
\end{table}


\section{Explanation of Results}
\label{sec:explanation-results-1}
The above access matrix shows who has access to what files from the given information.
\begin{enumerate}
\item Alice will have complete access to File1, which is engineering files to perform her job,
      read and write on File2, to log the timekeeping and also to read her timesheets. The access 
       on File2 also depends on how system is designed - to allow the reads on timesheets.
\item Bob will have complete access on File2, which is accounts.
      read-only access on File1 and File3 to audit the information.
\item Eve will have read and write on File2 to log her timekeeping and also to read her timesheets.
      read-only access on File1 and File3 to read designs and her customer(s) data.
\item Knowsitall has complete access on File3 and File4.
    read and execute on File1 and File2 to get the information to create product pricing.
    To maintain data integrity, though he is a CEO it is not required to have write access on 
    File1 (Engineering Files) and File2 (Account department).
\end{enumerate}

%%------ Question 2 -------%%

 \chapter{Question 2}
 \label{cha:ques2}
  
 \section{Problem Statement}
 \label{sec:problem-statement-2}

 Consider the access control matrix you generated and a concept of
 operations Using Higher Order Logic in which Eve, as Mr. Coyote’s
 customer rep, wants to create a new letter for Mr.  Coyote. Eve wants
 this letter to describe the wonderful things Acme’s new, custom
 design will do for him and how little it will cost him to get
 it. Derive the inference rule that controls this situation and prove
 it

\section{Relevant Code}
\label{sec:relevant-code-2}


\lstset{frameround=tttt}
\begin{lstlisting}[frame=tRBL]

1. Eve says <create letter, File2^File3>          Eve Request
2. Bob controls (Eve controls <read, File2>)      Access Policy
3. CRM controls (Eve controls <read, File3>)      Access Policy
4. ACL2 => Bob                                    Trust Assumption
5. ACL3 => CRM                                    Trust Assumption
6. ACL says Eve controls <read, File2> ^
            Eve controls <read, File3>            Access List
7. ACL says (Eve controls <read, File2 ^ File3)   6 simplify says
8. <create letter, File2^File3>                   1,2,3,4,5,7 ticket rule
   
\end{lstlisting}

\lstinputlisting{../HOL/question2Script.sml}


\section{Execution Transcripts}
\label{sec:exec-transcr-2}

\setcounter{sessioncount}{0}
\begin{session}
  \begin{scriptsize}
\begin{verbatim}


---------------------------------------------------------------------
       HOL-4 [Kananaskis 11 (stdknl, built Sat Aug 19 09:30:06 2017)]

       For introductory HOL help, type: help "hol";
       To exit type <Control>-D
---------------------------------------------------------------------
[extending loadPath with Holmakefile INCLUDES variable]
> > > > *** Globals.show_assums now true ***
> *** Globals.show_assums now false ***
> # # <<HOL message: Defined type: "commands">>
> # # <<HOL message: Defined type: "people">>
> # # <<HOL message: Defined type: "roles">>
> # # <<HOL message: Defined type: "keyPrinc">>
> # # <<HOL message: Defined type: "principals">>
> val question2Thm =
   |- (M,Oi,Os) sat Name (PR (Role owner)) controls prop grant ⇒
   (M,Oi,Os) sat
   reps (Name (PR (Staff Eve))) (Name (PR (Role owner))) (prop grant) ⇒
   (M,Oi,Os) sat
   Name (Key (Staff Eve)) quoting Name (PR (Role owner)) says
   prop grant ⇒
   (M,Oi,Os) sat prop grant impf prop deny ⇒
   (M,Oi,Os) sat
   Name (Key (Role requester)) speaks_for Name (PR (Role requester)) ⇒
   (M,Oi,Os) sat
   Name (Key (Role requester)) says
   Name (Key (Staff Eve)) speaks_for Name (PR (Staff Eve)) ⇒
   (M,Oi,Os) sat
   Name (PR (Role requester)) controls
   Name (Key (Staff Eve)) speaks_for Name (PR (Staff Eve)) ⇒
   (M,Oi,Os) sat
   Name (Key (Staff Bob)) quoting Name (PR (Role Operator)) says
   prop deny:
   thm
val it = (): unit
> 
*** Emacs/HOL command completed ***

>  

\end{verbatim}
  \end{scriptsize}
\end{session}

\subsection{Explanation of Results}
\label{sec:explanation-results-2}
The above results shows that the requirements are satisfied.


%%------ Question 3 -------%%

 \chapter{Question 3}
 \label{cha:exercise-3}
  
 \section{Problem Statement}
 \label{sec:problem-statement-3}

 Discuss if the derived inference rule has been proven and if so is
 adequate to answer Mr, Coyote’s concerns about security. If it is
 adequate to answer these concerns, you need to support that
 assertion. That is, you need to specifically say why it is
 adequate. In addition, briefly discuss if this access control system
 could result in Eve providing information to Acme’s competitors,
 which would enable them to work for or against Acme’s customers.

\section{Summary}
\label{sec:summary-3}
The derived inference rule was proved but it is not fully adequate to
answer the Mr. Coyote's concerns on the security of complete system.
Because this could answer on access to the files.  But even with the
read access how the system is restricting the data sharing to outside
needs to be addressed.

Also this access control system \textbf{could result} in Eve providing information to Acme’s
competitors, which would enable them to work for or against Acme’s
customers.  Because Eve has access to the Engineering files, where she
can get the new custom designs that are being worked by Engineering
teams and also the customers information.  So if Eve is a bad person,
she can read this information and share it to the competitors. This
can be refered as \textbf{internal threat}.

\subsection{Conclusion}
\label{sec:conclusion-3}
More detailed information is needed to answer the Coyote's question and this will also helps how the internal threats can be controlled.


\chapter{Appendix A: Question 2}
\label{cha:appendix-a:1}

The following code is from the file question2Script.sml
\lstinputlisting{../HOL/question2Script.sml}


\end{document}