\documentclass{report}
\title{Project 4}
\author{Bharath Karumudi}
\date{\today}

%%........................................%%
%% Loading the Packages 
%%........................................%%
\usepackage{634format}
\usepackage{enumerate}
\usepackage{listings}
\usepackage{amsmath}
\usepackage{hyperref}
\usepackage{holtex}
\usepackage{holtexbasic}
\usepackage{amssymb}
% =====================================================================
%
% Macros for typesetting the HOL system manual
%
% =====================================================================

% ---------------------------------------------------------------------
% Abbreviations for words and phrases
% ---------------------------------------------------------------------

\newcommand\TUTORIAL{{\footnotesize\sl TUTORIAL}}
\newcommand\DESCRIPTION{{\footnotesize\sl DESCRIPTION}}
\newcommand\REFERENCE{{\footnotesize\sl REFERENCE}}
\newcommand\LOGIC{{\footnotesize\sl LOGIC}}
\newcommand\LIBRARIES{{\footnotesize\sl LIBRARIES}}

\newcommand{\bs}{\texttt{\char'134}} % backslash
\newcommand{\lb}{\texttt{\char'173}} % left brace
\newcommand{\rb}{\texttt{\char'175}} % right brace
\newcommand{\td}{\texttt{\char'176}} % tilde
\newcommand{\lt}{\texttt{\char'74}} % less than
\newcommand{\gt}{\texttt{\char'76}} % greater than
\newcommand{\dol}{\texttt{\char'44}} % dollar
% double back quotes ``
\newcommand{\dq}{\texttt{\char'140\char'140}}
%These macros were included by slind:

\newcommand{\holquote}[1]{\dq#1\dq}

\def\HOL{{\small HOL}}
\def\holn{\HOL}  % i.e. hol n(inety-eight), no digits in
                 % macro names is a bit of a pain; deciding to do away
                 % with hol98 nomenclature means that we just want to
                 % write HOL for hol98.
\def\holnversion{Kananaskis-7}
\def\holnsversion{Kananaskis~7} % version with space rather than hyphen
\def\LCF{{\small LCF}}
\def\LCFLSM{{\small LCF{\kern-.2em}{\normalsize\_}{\kern0.1em}LSM}}
\def\PPL{{\small PP}{\kern-.095em}$\lambda$}
\def\PPLAMBDA{{\small PPLAMBDA}}
\def\ML{{\small ML}}
\def\holmake{\texttt{Holmake}}

\newcommand\ie{\mbox{i{.}e{.}}}
\newcommand\eg{\mbox{e{.}g{.}}}
\newcommand\viz{\mbox{viz{.}}}
\newcommand\adhoc{\mbox{\it ad hoc}}
\newcommand\etal{{\it et al.\/}}
\newcommand\etc{\mbox{etc{.}}}

% ---------------------------------------------------------------------
% Simple abbreviations and macros for mathematical typesetting
% ---------------------------------------------------------------------

\newcommand\fun{{\to}}
\newcommand\prd{{\times}}

\newcommand\conj{\ \wedge\ }
\newcommand\disj{\ \vee\ }
\newcommand\imp{ \Rightarrow }
\newcommand\eqv{\ \equiv\ }
\newcommand\cond{\rightarrow}
\newcommand\vbar{\mid}
\newcommand\turn{\ \vdash\ }
\newcommand\hilbert{\varepsilon}
\newcommand\eqdef{\ \equiv\ }

\newcommand\natnums{\mbox{${\sf N}\!\!\!\!{\sf N}$}}
\newcommand\bools{\mbox{${\sf T}\!\!\!\!{\sf T}$}}

\newcommand\p{$\prime$}
\newcommand\f{$\forall$\ }
\newcommand\e{$\exists$\ }

\newcommand\orr{$\vee$\ }
\newcommand\negg{$\neg$\ }

\newcommand\arrr{$\rightarrow$}
\newcommand\hex{$\sharp $}

\newcommand{\uquant}[1]{\forall #1.\ }
\newcommand{\equant}[1]{\exists #1.\ }
\newcommand{\hquant}[1]{\hilbert #1.\ }
\newcommand{\iquant}[1]{\exists ! #1.\ }
\newcommand{\lquant}[1]{\lambda #1.\ }

\newcommand{\leave}[1]{\\[#1]\noindent}
\newcommand\entails{\mbox{\rule{.3mm}{4mm}\rule[2mm]{.2in}{.3mm}}}

% ---------------------------------------------------------------------
% Font-changing commands
% ---------------------------------------------------------------------

\newcommand{\theory}[1]{\hbox{{\small\tt #1}}}
\newcommand{\theoryimp}[1]{\texttt{#1}}

\newcommand{\con}[1]{{\sf #1}}
\newcommand{\rul}[1]{{\tt #1}}
\newcommand{\ty}[1]{\textsl{#1}}

\newcommand{\ml}[1]{\mbox{{\def\_{\char'137}\texttt{#1}}}}
\newcommand{\holtxt}[1]{\ml{#1}}
\newcommand\ms{\tt}
\newcommand{\s}[1]{{\small #1}}

\newcommand{\pin}[1]{{\bf #1}}
\def\m#1{\mbox{\normalsize$#1$}}

% ---------------------------------------------------------------------
% Abbreviations for particular mathematical constants etc.
% ---------------------------------------------------------------------

\newcommand\T{\con{T}}
\newcommand\F{\con{F}}
\newcommand\OneOne{\con{One\_One}}
\newcommand\OntoSubset{\con{Onto\_Subset}}
\newcommand\Onto{\con{Onto}}
\newcommand\TyDef{\con{Type\_Definition}}
\newcommand\Inv{\con{Inv}}
\newcommand\com{\con{o}}
\newcommand\Id{\con{I}}
\newcommand\MkPair{\con{Mk\_Pair}}
\newcommand\IsPair{\con{Is\_Pair}}
\newcommand\Fst{\con{Fst}}
\newcommand\Snd{\con{Snd}}
\newcommand\Suc{\con{Suc}}
\newcommand\Nil{\con{Nil}}
\newcommand\Cons{\con{Cons}}
\newcommand\Hd{\con{Hd}}
\newcommand\Tl{\con{Tl}}
\newcommand\Null{\con{Null}}
\newcommand\ListPrimRec{\con{List\_Prim\_Rec}}


\newcommand\SimpRec{\con{Simp\_Rec}}
\newcommand\SimpRecRel{\con{Simp\_Rec\_Rel}}
\newcommand\SimpRecFun{\con{Simp\_Rec\_Fun}}
\newcommand\PrimRec{\con{Prim\_Rec}}
\newcommand\PrimRecRel{\con{Prim\_Rec\_Rel}}
\newcommand\PrimRecFun{\con{Prim\_Rec\_Fun}}

\newcommand\bool{\ty{bool}}
\newcommand\num{\ty{num}}
\newcommand\ind{\ty{ind}}
\newcommand\lst{\ty{list}}

% ---------------------------------------------------------------------
% \minipagewidth = \textwidth minus 1.02 em
% ---------------------------------------------------------------------

\newlength{\minipagewidth}
\setlength{\minipagewidth}{\textwidth}
\addtolength{\minipagewidth}{-1.02em}

% ---------------------------------------------------------------------
% Environment for the items on the title page of a case study
% ---------------------------------------------------------------------

\newenvironment{inset}[1]{\noindent{\large\bf #1}\begin{list}%
{}{\setlength{\leftmargin}{\parindent}%
\setlength{\topsep}{-.1in}}\item }{\end{list}\vskip .4in}

% ---------------------------------------------------------------------
% Macros for little HOL sessions displayed in boxes.
%
% Usage: (1) \setcounter{sessioncount}{1} resets the session counter
%
%        (2) \begin{session}\begin{verbatim}
%             .
%              < lines from hol session >
%             .
%            \end{verbatim}\end{session}
%
%            typesets the session in a numbered box.
% ---------------------------------------------------------------------

\newlength{\hsbw}
\setlength{\hsbw}{\textwidth}
\addtolength{\hsbw}{-\arrayrulewidth}
\addtolength{\hsbw}{-\tabcolsep}
\newcommand\HOLSpacing{13pt}

\newcounter{sessioncount}
\setcounter{sessioncount}{0}

\newenvironment{session}{\begin{flushleft}
 \refstepcounter{sessioncount}
 \begin{tabular}{@{}|c@{}|@{}}\hline
 \begin{minipage}[b]{\hsbw}
 \vspace*{-.5pt}
 \begin{flushright}
 \rule{0.01in}{.15in}\rule{0.3in}{0.01in}\hspace{-0.35in}
 \raisebox{0.04in}{\makebox[0.3in][c]{\footnotesize\sl \thesessioncount}}
 \end{flushright}
 \vspace*{-.55in}
 \begingroup\small\baselineskip\HOLSpacing}{\endgroup\end{minipage}\\ \hline
 \end{tabular}
 \end{flushleft}}

% ---------------------------------------------------------------------
% Macro for boxed ML functions, etc.
%
% Usage: (1) \begin{holboxed}\begin{verbatim}
%               .
%               < lines giving names and types of mk functions >
%               .
%            \end{verbatim}\end{holboxed}
%
%            typesets the given lines in a box.
%
%            Conventions: lines are left-aligned under the "g" of begin,
%            and used to highlight primary reference for the ml function(s)
%            that appear in the box.
% ---------------------------------------------------------------------

\newenvironment{holboxed}{\begin{flushleft}
  \begin{tabular}{@{}|c@{}|@{}}\hline
  \begin{minipage}[b]{\hsbw}
% \vspace*{-.55in}
  \vspace*{.06in}
  \begingroup\small\baselineskip\HOLSpacing}{\endgroup\end{minipage}\\ \hline
  \end{tabular}
  \end{flushleft}}

% ---------------------------------------------------------------------
% Macro for unboxed ML functions, etc.
%
% Usage: (1) \begin{hol}\begin{verbatim}
%               .
%               < lines giving names and types of mk functions >
%               .
%            \end{verbatim}\end{hol}
%
%            typesets the given lines exactly like {boxed}, except there's
%            no box.
%
%            Conventions: lines are left-aligned under the "g" of begin,
%            and used to display ML code in verbatim, left aligned.
% ---------------------------------------------------------------------

\newenvironment{hol}{\begin{flushleft}
 \begin{tabular}{c@{}@{}}
 \begin{minipage}[b]{\hsbw}
% \vspace*{-.55in}
 \vspace*{.06in}
 \begingroup\small\baselineskip\HOLSpacing}{\endgroup\end{minipage}\\
 \end{tabular}
 \end{flushleft}}

% ---------------------------------------------------------------------
% Emphatic brackets
% ---------------------------------------------------------------------

\newcommand\leb{\lbrack\!\lbrack}
\newcommand\reb{\rbrack\!\rbrack}


% ---------------------------------------------------------------------
% Quotations
% ---------------------------------------------------------------------


%These macros were included by ap; they are used in Chapters 9 and 10
%of the HOL DESCRIPTION

\newcommand{\inds}%standard infinite set
 {\mbox{\rm I}}

\newcommand{\ch}%standard choice function
 {\mbox{\rm ch}}

\newcommand{\den}[1]%denotational brackets
 {[\![#1]\!]}

\newcommand{\two}%standard 2-element set
 {\mbox{\rm 2}}

\newcommand{\HOLexerciseNineDate}{02 August 2019}
\newcommand{\HOLexerciseNineTime}{20:36}
\begin{SaveVerbatim}{HOLexerciseNineTheoremsabsorptionRule}
\HOLTokenTurnstile{} \HOLSymConst{\HOLTokenForall{}}\HOLBoundVar{p} \HOLBoundVar{q}. (\HOLBoundVar{p} \HOLSymConst{\HOLTokenImp{}} \HOLBoundVar{q}) \HOLSymConst{\HOLTokenImp{}} \HOLBoundVar{p} \HOLSymConst{\HOLTokenImp{}} \HOLBoundVar{p} \HOLSymConst{\HOLTokenConj{}} \HOLBoundVar{q}
\end{SaveVerbatim}
\newcommand{\HOLexerciseNineTheoremsabsorptionRule}{\UseVerbatim{HOLexerciseNineTheoremsabsorptionRule}}
\begin{SaveVerbatim}{HOLexerciseNineTheoremsabsorptionRuleTwo}
\HOLTokenTurnstile{} \HOLSymConst{\HOLTokenForall{}}\HOLBoundVar{p} \HOLBoundVar{q} \HOLBoundVar{r} \HOLBoundVar{s}. (\HOLBoundVar{p} \HOLSymConst{\HOLTokenImp{}} \HOLBoundVar{q}) \HOLSymConst{\HOLTokenConj{}} (\HOLBoundVar{r} \HOLSymConst{\HOLTokenImp{}} \HOLBoundVar{s}) \HOLSymConst{\HOLTokenImp{}} \HOLBoundVar{p} \HOLSymConst{\HOLTokenDisj{}} \HOLBoundVar{r} \HOLSymConst{\HOLTokenImp{}} \HOLBoundVar{q} \HOLSymConst{\HOLTokenDisj{}} \HOLBoundVar{s}
\end{SaveVerbatim}
\newcommand{\HOLexerciseNineTheoremsabsorptionRuleTwo}{\UseVerbatim{HOLexerciseNineTheoremsabsorptionRuleTwo}}
\begin{SaveVerbatim}{HOLexerciseNineTheoremsconstructiveDilemmaRule}
\HOLTokenTurnstile{} \HOLSymConst{\HOLTokenForall{}}\HOLBoundVar{p} \HOLBoundVar{q} \HOLBoundVar{r} \HOLBoundVar{s}. (\HOLBoundVar{p} \HOLSymConst{\HOLTokenImp{}} \HOLBoundVar{q}) \HOLSymConst{\HOLTokenConj{}} (\HOLBoundVar{r} \HOLSymConst{\HOLTokenImp{}} \HOLBoundVar{s}) \HOLSymConst{\HOLTokenImp{}} \HOLBoundVar{p} \HOLSymConst{\HOLTokenDisj{}} \HOLBoundVar{r} \HOLSymConst{\HOLTokenImp{}} \HOLBoundVar{q} \HOLSymConst{\HOLTokenDisj{}} \HOLBoundVar{s}
\end{SaveVerbatim}
\newcommand{\HOLexerciseNineTheoremsconstructiveDilemmaRule}{\UseVerbatim{HOLexerciseNineTheoremsconstructiveDilemmaRule}}
\begin{SaveVerbatim}{HOLexerciseNineTheoremsconstructiveDilemmaRuleTwo}
\HOLTokenTurnstile{} \HOLSymConst{\HOLTokenForall{}}\HOLBoundVar{p} \HOLBoundVar{q} \HOLBoundVar{r} \HOLBoundVar{s}. (\HOLBoundVar{p} \HOLSymConst{\HOLTokenImp{}} \HOLBoundVar{q}) \HOLSymConst{\HOLTokenConj{}} (\HOLBoundVar{r} \HOLSymConst{\HOLTokenImp{}} \HOLBoundVar{s}) \HOLSymConst{\HOLTokenImp{}} \HOLBoundVar{p} \HOLSymConst{\HOLTokenDisj{}} \HOLBoundVar{r} \HOLSymConst{\HOLTokenImp{}} \HOLBoundVar{q} \HOLSymConst{\HOLTokenDisj{}} \HOLBoundVar{s}
\end{SaveVerbatim}
\newcommand{\HOLexerciseNineTheoremsconstructiveDilemmaRuleTwo}{\UseVerbatim{HOLexerciseNineTheoremsconstructiveDilemmaRuleTwo}}
\newcommand{\HOLexerciseNineTheorems}{
\HOLThmTag{exercise9}{absorptionRule}\HOLexerciseNineTheoremsabsorptionRule
\HOLThmTag{exercise9}{absorptionRule2}\HOLexerciseNineTheoremsabsorptionRuleTwo
\HOLThmTag{exercise9}{constructiveDilemmaRule}\HOLexerciseNineTheoremsconstructiveDilemmaRule
\HOLThmTag{exercise9}{constructiveDilemmaRule2}\HOLexerciseNineTheoremsconstructiveDilemmaRuleTwo
}

\newcommand{\HOLexerciseOneZeroDate}{02 August 2019}
\newcommand{\HOLexerciseOneZeroTime}{20:36}
\begin{SaveVerbatim}{HOLexerciseOneZeroTheoremsproblemonethm}
\HOLTokenTurnstile{} \HOLFreeVar{M} \HOLFreeVar{s}
\end{SaveVerbatim}
\newcommand{\HOLexerciseOneZeroTheoremsproblemonethm}{\UseVerbatim{HOLexerciseOneZeroTheoremsproblemonethm}}
\begin{SaveVerbatim}{HOLexerciseOneZeroTheoremsproblemtwothm}
\HOLTokenTurnstile{} \HOLFreeVar{p} \HOLSymConst{\HOLTokenImp{}} \HOLSymConst{\HOLTokenNeg{}}\HOLFreeVar{q}
\end{SaveVerbatim}
\newcommand{\HOLexerciseOneZeroTheoremsproblemtwothm}{\UseVerbatim{HOLexerciseOneZeroTheoremsproblemtwothm}}
\newcommand{\HOLexerciseOneZeroTheorems}{
\HOLThmTag{exercise10}{problemonethm}\HOLexerciseOneZeroTheoremsproblemonethm
\HOLThmTag{exercise10}{problemtwothm}\HOLexerciseOneZeroTheoremsproblemtwothm
}

%.........................................%%
%.........................................%%

\begin{document}
 \lstset{language=ML}
 \maketitle{}

 \begin{abstract}
   This project is to demonstrate the capabilities of implementing
   constructing and deconstructing HOL Terms using the tools and
   techniques - \LaTeX{}, AcuTeX, emacs and ML. 

   Each chapter documents the given problems with a structure of:
   \begin{enumerate}
   \item Problem Statement
   \item Relevant Code
   \item Execution Transcripts
   \item Explanation of results
   \end{enumerate}

 \end{abstract}


 \begin{acknowledgments}
  Professor Marvine Hamner and Professor Shiu-Kai Chin who taught the
  Certified Security By Design.
 \end{acknowledgments}

 \tableofcontents{}

 \chapter{Executive Summary}
 \label{cha:executive-summary}

\textbf{Some requirements for this project are statisfied specifically,}
 and by using HOL proved the below theorems:
\begin{quote}
\HOLexerciseNineTheorems
\HOLexerciseOneZeroTheorems
\end{quote}
10.4.3 is not included.

%%------ Exercise 9.5.1 -------%%

 \chapter{Exercise 9.5.1}
 \label{cha:exercise-9.5.1}
  
 \section{Problem Statement}
 \label{sec:problem-statement-1}

In this exercise we need to prove the theorem:
\HOLTokenTurnstile{} \HOLSymConst{\HOLTokenForall{}}\HOLBoundVar{p} \HOLBoundVar{q}. (\HOLBoundVar{p} \HOLSymConst{\HOLTokenImp{}} \HOLBoundVar{q}) \HOLSymConst{\HOLTokenImp{}} \HOLBoundVar{p} \HOLSymConst{\HOLTokenImp{}} \HOLBoundVar{p} \HOLSymConst{\HOLTokenConj{}} \HOLBoundVar{q}

\section{Relevant Code}
\label{sec:relevant-code-1}
\lstinputlisting{HOLCode/ex-9-5-1.sml}

\section{Execution Transcripts}
\label{sec:exec-transcr-1}

\setcounter{sessioncount}{0}
\begin{session}
  \begin{scriptsize}
\begin{verbatim}

---------------------------------------------------------------------
       HOL-4 [Kananaskis 11 (stdknl, built Sat Aug 19 09:30:06 2017)]

       For introductory HOL help, type: help "hol";
       To exit type <Control>-D
---------------------------------------------------------------------
> > > > # # # # # # # # # ** types trace now on
> *** Globals.show_assums now true ***
> # # # # # # # # # ** Unicode trace now off
> 
> # # # # # val absorptionRule =
    [] |- !(p :bool) (q :bool). (p ==> q) ==> p ==> p /\ q:
   thm
>  
\end{verbatim}
  \end{scriptsize}
\end{session}

\subsection{Explanation of Results}
\label{sec:explanation-results-1}
The above results shows that the requirements are satisfied.

%%------ Exercise 9.5.2 -------%%

 \chapter{Exercise 9.5.2}
 \label{cha:exercise-9.5.2}
  
 \section{Problem Statement}
 \label{sec:problem-statement-2}

In this exercise we need to prove the theorem:
\HOLTokenTurnstile{} \HOLSymConst{\HOLTokenForall{}}\HOLBoundVar{p} \HOLBoundVar{q} \HOLBoundVar{r} \HOLBoundVar{s}. (\HOLBoundVar{p} \HOLSymConst{\HOLTokenImp{}} \HOLBoundVar{q}) \HOLSymConst{\HOLTokenConj{}} (\HOLBoundVar{r} \HOLSymConst{\HOLTokenImp{}} \HOLBoundVar{s}) \HOLSymConst{\HOLTokenImp{}} \HOLBoundVar{p} \HOLSymConst{\HOLTokenDisj{}} \HOLBoundVar{r} \HOLSymConst{\HOLTokenImp{}} \HOLBoundVar{q} \HOLSymConst{\HOLTokenDisj{}} \HOLBoundVar{s}

\section{Relevant Code}
\label{sec:relevant-code-2}
\lstinputlisting{HOLCode/ex-9-5-2.sml}

\section{Execution Transcripts}
\label{sec:exec-transcr-2}

\setcounter{sessioncount}{0}
\begin{session}
  \begin{scriptsize}
\begin{verbatim}

---------------------------------------------------------------------
       HOL-4 [Kananaskis 11 (stdknl, built Sat Aug 19 09:30:06 2017)]

       For introductory HOL help, type: help "hol";
       To exit type <Control>-D
---------------------------------------------------------------------
> > > > # # # # # # # # # ** types trace now on
> *** Globals.show_assums now true ***
> # # # # # # # # # ** Unicode trace now off
> 
> # # # # # # # # # val constructiveDilemmaRule =
    []
|- !(p :bool) (q :bool) (r :bool) (s :bool).
     (p ==> q) /\ (r ==> s) ==> p \/ r ==> q \/ s:
   thm
> 
> 

\end{verbatim}
  \end{scriptsize}
\end{session}

\subsection{Explanation of Results}
\label{sec:explanation-results-2}
The above results shows that the requirements are satisfied.


%%------ Exercise 9.5.3 -------%%

 \chapter{Exercise 9.5.3}
 \label{cha:exercise-9.5.3}
  
 \section{Problem Statement}
 \label{sec:problem-statement-3}

In this exercise we need to prove the theorem:  

\HOLTokenTurnstile{} \HOLSymConst{\HOLTokenForall{}}\HOLBoundVar{p} \HOLBoundVar{q} \HOLBoundVar{r} \HOLBoundVar{s}. (\HOLBoundVar{p} \HOLSymConst{\HOLTokenImp{}} \HOLBoundVar{q}) \HOLSymConst{\HOLTokenConj{}} (\HOLBoundVar{r} \HOLSymConst{\HOLTokenImp{}} \HOLBoundVar{s}) \HOLSymConst{\HOLTokenImp{}} \HOLBoundVar{p} \HOLSymConst{\HOLTokenDisj{}} \HOLBoundVar{r} \HOLSymConst{\HOLTokenImp{}} \HOLBoundVar{q} \HOLSymConst{\HOLTokenDisj{}} \HOLBoundVar{s}

\HOLTokenTurnstile{} \HOLSymConst{\HOLTokenForall{}}\HOLBoundVar{p} \HOLBoundVar{q} \HOLBoundVar{r} \HOLBoundVar{s}. (\HOLBoundVar{p} \HOLSymConst{\HOLTokenImp{}} \HOLBoundVar{q}) \HOLSymConst{\HOLTokenConj{}} (\HOLBoundVar{r} \HOLSymConst{\HOLTokenImp{}} \HOLBoundVar{s}) \HOLSymConst{\HOLTokenImp{}} \HOLBoundVar{p} \HOLSymConst{\HOLTokenDisj{}} \HOLBoundVar{r} \HOLSymConst{\HOLTokenImp{}} \HOLBoundVar{q} \HOLSymConst{\HOLTokenDisj{}} \HOLBoundVar{s}

\section{Relevant Code}
\label{sec:relevant-code-3}
\lstinputlisting{HOLCode/ex-9-5-3.sml}

\section{Execution Transcripts}
\label{sec:exec-transcr-3}

\setcounter{sessioncount}{0}
\begin{session}
  \begin{scriptsize}
\begin{verbatim}

---------------------------------------------------------------------
       HOL-4 [Kananaskis 11 (stdknl, built Sat Aug 19 09:30:06 2017)]

       For introductory HOL help, type: help "hol";
       To exit type <Control>-D
---------------------------------------------------------------------
> > > > # # # # # # # # # ** types trace now on
> *** Globals.show_assums now true ***
> # # # # # # # # # ** Unicode trace now off
> 
> # # # # Meson search level: ................
val absorptionRule2 =
    []
|- !(p :bool) (q :bool) (r :bool) (s :bool).
     (p ==> q) /\ (r ==> s) ==> p \/ r ==> q \/ s:
   thm
> 
> # # # # Meson search level: ................
val constructiveDilemmaRule2 =
    []
|- !(p :bool) (q :bool) (r :bool) (s :bool).
     (p ==> q) /\ (r ==> s) ==> p \/ r ==> q \/ s:
   thm
> 
> 
\end{verbatim}
  \end{scriptsize}
\end{session}

\subsection{Explanation of Results}
\label{sec:explanation-results-3}
The above results shows that the requirements are satisfied.



%%------ Exercise 10.4.1 -------%%

 \chapter{Exercise 10.4.1}
 \label{cha:exercise-10.4.1}
  
 \section{Problem Statement}
 \label{sec:problem-statement-4}

In this exercise we need to prove the theorem:
\HOLTokenTurnstile{} \HOLFreeVar{M} \HOLFreeVar{s}

\section{Relevant Code}
\label{sec:relevant-code-4}
\lstinputlisting{HOLCode/ex-10-4-1.sml}

\section{Execution Transcripts}
\label{sec:exec-transcr-4}

\setcounter{sessioncount}{0}
\begin{session}
  \begin{scriptsize}
\begin{verbatim}

---------------------------------------------------------------------
       HOL-4 [Kananaskis 11 (stdknl, built Sat Aug 19 09:30:06 2017)]

       For introductory HOL help, type: help "hol";
       To exit type <Control>-D
---------------------------------------------------------------------
> > > > # # # # # # # # # ** types trace now on
> *** Globals.show_assums now true ***
> # # # # # # # # # ** Unicode trace now off
> 
> # # # # # val problemonethm =
   
[(P :'a -> bool) (s :'a),
 !(x :'a). (P :'a -> bool) x ==> (M :'a -> bool) x]
|- (M :'a -> bool) (s :'a):
   thm
> > 
> 
\end{verbatim}
  \end{scriptsize}
\end{session}

\subsection{Explanation of Results}
\label{sec:explanation-results-4}
The above results shows that the requirements are satisfied.


%%------ Exercise 10.4.2 -------%%

 \chapter{Exercise 10.4.2}
 \label{cha:exercise-10.4.2}
  
 \section{Problem Statement}
 \label{sec:problem-statement-5}

In this exercise we need to prove the theorem:
\HOLTokenTurnstile{} \HOLFreeVar{p} \HOLSymConst{\HOLTokenImp{}} \HOLSymConst{\HOLTokenNeg{}}\HOLFreeVar{q}

\section{Relevant Code}
\label{sec:relevant-code-5}
\lstinputlisting{HOLCode/ex-10-4-2.sml}

\section{Execution Transcripts}
\label{sec:exec-transcr-5}

\setcounter{sessioncount}{0}
\begin{session}
  \begin{scriptsize}
\begin{verbatim}

---------------------------------------------------------------------
       HOL-4 [Kananaskis 11 (stdknl, built Sat Aug 19 09:30:06 2017)]

       For introductory HOL help, type: help "hol";
       To exit type <Control>-D
---------------------------------------------------------------------
> > > > # # # # # # # # # ** types trace now on
> *** Globals.show_assums now true ***
> # # # # # # # # # ** Unicode trace now off
> 
> # # # # # # # # # # # # # # # # val problemtwothm =
   
[~(s :bool), (r :bool) ==> (s :bool),
 (p :bool) /\ (q :bool) ==> (r :bool)] |- (p :bool) ==> ~(q :bool):
   thm
> 
*** Emacs/HOL command completed ***

> 
> 
\end{verbatim}
  \end{scriptsize}
\end{session}

\subsection{Explanation of Results}
\label{sec:explanation-results-5}
The above results shows that the requirements are satisfied.

%%------ Exercise 10.4.3 -------%%
%% --- Not solved --- %%

\chapter{Appendix A: Chapter 9}
\label{cha:appendix-a:chapter9}

The following code is from the file project3bScript.sml
\lstinputlisting{../HOL/exercise9Script.sml}

\chapter{Appendix B: Chapter 10}
\label{cha:appendix-a:chapter10}

The following code is from the file project3bScript.sml
\lstinputlisting{../HOL/exercise10Script.sml}

\end{document}