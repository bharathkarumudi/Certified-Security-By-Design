\documentclass{article}

\title{Project A - An Introduction to LaTex and Me}
\author{Bharath Karumudi}
\date{\today}

\begin{document}

\maketitle{}

\begin{abstract}
  The purpose of this document is for Project A, to produce a basic
  technical report of professional quality using LaTex. The contents
  of the document includes: (1) Where I grew up, (2) My program of
  study, (3) Why I picked my program of study, (4) Why I am taking
  Assurance fundamentals and (5) What I hope to learn in Assurance
  fundamentals.

\end{abstract}

\textbf{Acknowledgements}: Professor Marvine Hamner and Professor Shiu-Kai Chin who helped in usage of LaTex.

\section{Where I Grew Up}
\label{sec:where-i-grew}

\begin{description}
\item[My home town] I grew up in a town of Andhra Pradesh, which is part of India.
\item[What I liked] I like the beach and also having all friends and family in the same town.
\item[What I disliked] The humid weather.

\end{description}

\section{My Program of Study}
\label{sec:my-program-study}
I enrolled in Masters in Cybersecurity program at Syracuse University in the fall of 2018.

\section{Why I Picked My Program of Study}
\label{sec:why-i-picked}
I am working in IT industry from last 8 years, but with a application background. With the technology evolving every day, I feel security is very much important to protect the assets. So I wanted to learn the Cybersecurity and work in security domain.

\section{Why I am Taking Assurance Fundamentals}
\label{sec:why-i-am}
\begin{itemize}
\item To understand the access controls.
\item To build secure environments.
\end{itemize}

\section{What I Hope to Learn in Assurance Fundamentals}
\label{sec:what-i-hope}
\begin{enumerate}
\item A practical way of implemeting the security.
\item Right way of doing the access controls, security and trust.
\end{enumerate}

\section{Executive Summary}
\label{sec:executive-summary}
\textbf{Summary}: All requirements are satisfied without exception and documented.

\end{document}
